%%
%% Beginning of file 'sample62.tex'
%%
%% Modified 2018 January
%%
%% This is a sample manuscript marked up using the
%% AASTeX v6.2 LaTeX 2e macros.
%%
%% AASTeX is now based on Alexey Vikhlinin's emulateapj.cls 
%% (Copyright 2000-2015).  See the classfile for details.

%% AASTeX requires revtex4-1.cls (http://publish.aps.org/revtex4/) and
%% other external packages (latexsym, graphicx, amssymb, longtable, and epsf).
%% All of these external packages should already be present in the modern TeX 
%% distributions.  If not they can also be obtained at www.ctan.org.

%% The first piece of markup in an AASTeX v6.x document is the \documentclass
%% command. LaTeX will ignore any data that comes before this command. The 
%% documentclass can take an optional argument to modify the output style.
%% The command below calls the preprint style  which will produce a tightly 
%% typeset, one-column, single-spaced document.  It is the default and thus
%% does not need to be explicitly stated.
%%
%%
%% using aastex version 6.2
\documentclass{aastex62}

%% The default is a single spaced, 10 point font, single spaced article.
%% There are 5 other style options available via an optional argument. They
%% can be envoked like this:
%%
%% \documentclass[argument]{aastex62}
%% 
%% where the layout options are:
%%
%%  twocolumn   : two text columns, 10 point font, single spaced article.
%%                This is the most compact and represent the final published
%%                derived PDF copy of the accepted manuscript from the publisher
%%  manuscript  : one text column, 12 point font, double spaced article.
%%  preprint    : one text column, 12 point font, single spaced article.  
%%  preprint2   : two text columns, 12 point font, single spaced article.
%%  modern      : a stylish, single text column, 12 point font, article with
%% 		  wider left and right margins. This uses the Daniel
%% 		  Foreman-Mackey and David Hogg design.
%%  RNAAS       : Preferred style for Research Notes which are by design 
%%                lacking an abstract and brief. DO NOT use \begin{abstract}
%%                and \end{abstract} with this style.
%%
%% Note that you can submit to the AAS Journals in any of these 6 styles.
%%
%% There are other optional arguments one can envoke to allow other stylistic
%% actions. The available options are:
%%
%%  astrosymb    : Loads Astrosymb font and define \astrocommands. 
%%  tighten      : Makes baselineskip slightly smaller, only works with 
%%                 the twocolumn substyle.
%%  times        : uses times font instead of the default
%%  linenumbers  : turn on lineno package.
%%  trackchanges : required to see the revision mark up and print its output
%%  longauthor   : Do not use the more compressed footnote style (default) for 
%%                 the author/collaboration/affiliations. Instead print all
%%                 affiliation information after each name. Creates a much
%%                 long author list but may be desirable for short author papers
%%
%% these can be used in any combination, e.g.
%%
%% \documentclass[twocolumn,linenumbers,trackchanges]{aastex62}
%%
%% AASTeX v6.* now includes \hyperref support. While we have built in specific
%% defaults into the classfile you can manually override them with the
%% \hypersetup command. For example,
%%
%%\hypersetup{linkcolor=red,citecolor=green,filecolor=cyan,urlcolor=magenta}
%%
%% will change the color of the internal links to red, the links to the
%% bibliography to green, the file links to cyan, and the external links to
%% magenta. Additional information on \hyperref options can be found here:
%% https://www.tug.org/applications/hyperref/manual.html#x1-40003
%%
%% If you want to create your own macros, you can do so
%% using \newcommand. Your macros should appear before
%% the \begin{document} command.
%%
\newcommand{\vdag}{(v)^\dagger}
\newcommand\aastex{AAS\TeX}
\newcommand\latex{La\TeX}

%% Reintroduced the \received and \accepted commands from AASTeX v5.2
\received{January 1, 2018}
\revised{January 7, 2018}
\accepted{\today}
%% Command to document which AAS Journal the manuscript was submitted to.
%% Adds "Submitted to " the arguement.
\submitjournal{ApJ}

%% Mark up commands to limit the number of authors on the front page.
%% Note that in AASTeX v6.2 a \collaboration call (see below) counts as
%% an author in this case.
%
%\AuthorCollaborationLimit=3
%
%% Will only show Schwarz, Muench and "the AAS Journals Data Scientist 
%% collaboration" on the front page of this example manuscript.
%%
%% Note that all of the author will be shown in the published article.
%% This feature is meant to be used prior to acceptance to make the
%% front end of a long author article more manageable. Please do not use
%% this functionality for manuscripts with less than 20 authors. Conversely,
%% please do use this when the number of authors exceeds 40.
%%
%% Use \allauthors at the manuscript end to show the full author list.
%% This command should only be used with \AuthorCollaborationLimit is used.

%% The following command can be used to set the latex table counters.  It
%% is needed in this document because it uses a mix of latex tabular and
%% AASTeX deluxetables.  In general it should not be needed.
%\setcounter{table}{1}

%%%%%%%%%%%%%%%%%%%%%%%%%%%%%%%%%%%%%%%%%%%%%%%%%%%%%%%%%%%%%%%%%%%%%%%%%%%%%%%%
%%
%% The following section outlines numerous optional output that
%% can be displayed in the front matter or as running meta-data.
%%
%% If you wish, you may supply running head information, although
%% this information may be modified by the editorial offices.
\shorttitle{Compact Binaries of BS Twins from Stellar Triples}
\shortauthors{Leigh \& Portegies Zwart}
%%
%% You can add a light gray and diagonal water-mark to the first page 
%% with this command:
% \watermark{text}
%% where "text", e.g. DRAFT, is the text to appear.  If the text is 
%% long you can control the water-mark size with:
%  \setwatermarkfontsize{dimension}
%% where dimension is any recognized LaTeX dimension, e.g. pt, in, etc.
%%
%%%%%%%%%%%%%%%%%%%%%%%%%%%%%%%%%%%%%%%%%%%%%%%%%%%%%%%%%%%%%%%%%%%%%%%%%%%%%%%%

%% This is the end of the preamble.  Indicate the beginning of the
%% manuscript itself with \begin{document}.

\begin{document}

\title{A Triple Origin for Twin Blue Stragglers in Compact Binaries}

%% LaTeX will automatically break titles if they run longer than
%% one line. However, you may use \\ to force a line break if
%% you desire. In v6.2 you can include a footnote in the title.

%% A significant change from earlier AASTEX versions is in the structure for 
%% calling author and affilations. The change was necessary to implement 
%% autoindexing of affilations which prior was a manual process that could 
%% easily be tedious in large author manuscripts.
%%
%% The \author command is the same as before except it now takes an optional
%% arguement which is the 16 digit ORCID. The syntax is:
%% \author[xxxx-xxxx-xxxx-xxxx]{Author Name}
%%
%% This will hyperlink the author name to the author's ORCID page. Note that
%% during compilation, LaTeX will do some limited checking of the format of
%% the ID to make sure it is valid.
%%
%% Use \affiliation for affiliation information. The old \affil is now aliased
%% to \affiliation. AASTeX v6.2 will automatically index these in the header.
%% When a duplicate is found its index will be the same as its previous entry.
%%
%% Note that \altaffilmark and \altaffiltext have been removed and thus 
%% can not be used to document secondary affiliations. If they are used latex
%% will issue a specific error message and quit. Please use multiple 
%% \affiliation calls for to document more than one affiliation.
%%
%% The new \altaffiliation can be used to indicate some secondary information
%% such as fellowships. This command produces a non-numeric footnote that is
%% set away from the numeric \affiliation footnotes.  NOTE that if an
%% \altaffiliation command is used it must come BEFORE the \affiliation call,
%% right after the \author command, in order to place the footnotes in
%% the proper location.
%%
%% Use \email to set provide email addresses. Each \email will appear on its
%% own line so you can put multiple email address in one \email call. A new
%% \correspondingauthor command is available in V6.2 to identify the
%% corresponding author of the manuscript. It is the author's responsibility
%% to make sure this name is also in the author list.
%%
%% While authors can be grouped inside the same \author and \affiliation
%% commands it is better to have a single author for each. This allows for
%% one to exploit all the new benefits and should make book-keeping easier.
%%
%% If done correctly the peer review system will be able to
%% automatically put the author and affiliation information from the manuscript
%% and save the corresponding author the trouble of entering it by hand.

\correspondingauthor{Nathan W. C. Leigh}
\email{nleigh@amnh.org}

\author{Nathan W. C. Leigh}
\affil{American Museum of Natural History \\
Department of Astrophysics \\
79th Street at Central Park West \\
New York, NY 10024-5192, USA}
\affil{Stony Brook University \\
Department of Physics and Astronomy\\
Stony Brook, NY 11794-3800, USA}
\affil{Departamento de Astronom\'ia \\ 
Facultad de Ciencias F\'isicas y Matem\'aticas \\ 
Universidad de Concepci\'on \\ 
Concepci\'on, Chile}


\author{Simon Portegies Zwart}
\affiliation{Leiden Observatory \\
Leiden University \\
PO Box 9513, 2300 RA \\
Leiden, the Netherlands}


%% Note that the \and command from previous versions of AASTeX is now
%% depreciated in this version as it is no longer necessary. AASTeX 
%% automatically takes care of all commas and "and"s between authors names.

%% AASTeX 6.2 has the new \collaboration and \nocollaboration commands to
%% provide the collaboration status of a group of authors. These commands 
%% can be used either before or after the list of corresponding authors. The
%% argument for \collaboration is the collaboration identifier. Authors are
%% encouraged to surround collaboration identifiers with ()s. The 
%% \nocollaboration command takes no argument and exists to indicate that
%% the nearby authors are not part of surrounding collaborations.

%\newcommand{\mnras}{\textrm{MNRAS}}
%\newcommand{\apj}{\textrm{ApJ}}
%\newcommand{\aap}{\textrm{A\&A}}
%\newcommand{\apjl}{\textrm{ApJ}}

\begin{abstract}
In this Letter, we propose a potential formation mechanism for twin blue stragglers in compact binaries that involves mass transfer from an evolved outer tertiary companion on to the inner binary via a circumbinary disk.  We apply our hypothetical scenario to the observed double blue straggler system Binary 7782 in the old open cluster NGC 188, and show that its observed properties are easily and even naturally reproduced within the context our proposed model.  Within the context of the hypothetical formation mechanism proposed here, the presented work predicts the following properties for the post-mass transfer double BS tertiary:  (1)  For the outer tertiary orbit, the initial orbital period should lie between 220 days $\lesssim$ P$_{\rm out}$ $\lesssim$ 1100 days, assuming initial masses for the inner binary components of m$_{\rm 1} =$ 1.1 M$_{\odot}$ and m$_{\rm 2} =$ 0.9 M$_{\odot}$ and an outer tertiary mass of m$_{\rm 3} =$ 1.4 M$_{\odot}$. (2)  Larger final WD masses, and hence core masses for the donor at the time of mass transfer, should correspond to larger final outer orbital periods for the tertiary. (3)  For the inner binary, the rotational axes of the BSs should be aligned with each other and the orbital plane of the outer tertiary WD. (4)  The BSs in the inner binary should have roughly equal masses, independent of their initial masses.  This predicts that the initially lower mass MS star should accrete the most, and should hence be polluted more significantly by any accreted material.  This could be observable in the surface layers of a radiative star (i.e., He, C and O if the donor is an RGB star, and/or s-process elements if the donor is an AGB star). (5) Twin BSs in compact binaries formed from the proposed mechanism should be more frequent in younger clusters with ages $\lesssim$ 4-6 Gyr, since the donor will have  radiative envelope.

\end{abstract}

\keywords{stars: blue stragglers -- binaries: general -- globular clusters: general -- scattering}


%%%%%%%%%%%%%%%%%%%%%%%%%%%%%%%
\section{Introduction} \label{intro}

Blue straggler (BS) stars appear brighter and bluer than the main-sequence turn-off (MSTO) in a cluster colour-magnitude diagram (CMD) \citep[e.g.][]{simunovic14,simunovic16}.  Two primary channels for BS formation have been proposed; mass transfer from an evolved donor on to a main-sequence star in a binary star system \citep[e.g.][]{mccrea64,knigge09,mathieu09,leigh11,geller11,geller12,gosnell14,gosnell15}, and direct stellar collisions involving main-sequence stars mediated via direct interactions involving binaries \citep[e.g.][]{hills75,shara97,leigh07,leigh11,leigh13,hypki13}.  Other possible, albeit related, formation mechanisms include mergers of compact MS-MS binaries, and mergers of the inner MS-MS binaries of hierarchical triple star systems induced by Lidov-Kozai oscillations coupled with tidal damping \citep[e.g.][]{perets09}.  

In spite of these specific predictions for the expected properties of BSs formed from each of the above production mechanisms, there exist many BSs with observed properties that defy these simple scenarios.  For example, in the old open cluster M67, there lurks a candidate triple system that is posited to host \textit{two} BSs \citep{vandenberg01,sandquist03}.  The observations suggest that the outer tertiary is itself a BS, with a mass $\sim$ 1.7 M$_{\odot}$ and orbiting the inner binary with a period of 1188.5 days \citep{sandquist03}.  The inner binary has a period of only 1.068 days \citep{vandenberg01}, and hosts a BS of mass 2.52 M$_{\odot}$.  Thus, in order to form this system after convolving its observed properties with its host cluster age, we require at least five stars in order to conserve the total system mass \citep{leigh11}.  As explained in \citet{leigh11}, this is strongly indicative of a dynamical origin for the system, and a single direct interaction involving a binary and a triple that resulted in two separate collision events is the most probable explanation for its origin (i.e., a single interaction involving two multiples with two or more stars is a more likely scenario to produce this system than two back-to-back direct binary-binary interactions).  

Even more curious, there exists in the old open cluster NGC 188 a double BS binary, called Binary 7782.  Specifically, \citet{mathieu09} observed a compact and mildly eccentric (i.e., e $\sim$ 0.1) binary star system with an orbital period of $\sim$ 10 days hosting \textit{two} blue stragglers.  During a given binary-binary interaction, the probability that not one but \textit{two} direct MS-MS collisions occur is less than a percent \citep{leonard89,leigh11,leigh12}.  Typically, such binaries are very wide with long orbital periods.  Thus, dynamically, it is very difficult to form a compact binary composed of two collision products during a single direct interaction in a star cluster.  So, how did Binary 7782 form?  MENTION BIMODAL P-E DISTRIBUTION, AND HOW THIS COULD BE SUGGESTIVE OF A TRIPLE ORIGIN FOR SOME OF THE SUBSET OF OBSERVED BSs.

In this Letter, we propose a potential formation channel for Binary 7782, and compact double BS binaries in general, which involves mass transfer from an outer tertiary companion on to an inner MS-MS binary.  In section~\ref{dyn}, we constrain the range of initial (i.e., pre-mass transfer) orbital parameters for a hypothetical outer tertiary companion, using a combination of dynamical and stellar evolution-based constraints.  In Section~\ref{sims} we present the numerical simulations used to study the mass transfer process in our hypothetical triple system, computed using the \texttt{AMUSE} software package, in order to study the evolution of the inner and outer orbital parameters during mass transfer.  We summarize and discuss the implications of our results for compact double BS binaries and, more generally, mass transfer in stellar triples in Section~\ref{discussion}.  

\section{Constraints on the present-day orbital parameters for a hypothesized tertiary companion in the compact BS Binary 7782} \label{dyn}

Consider a hierarchical triple system with component masses m$_{\rm 1}$ and m$_{\rm 2}$ for the inner binary, and mass m$_{\rm 3}$ for the outer tertiary companion.  The inner and outer binary orbital semi-major axes are denoted a$_{\rm in}$ and a$_{\rm out}$, respectively.  We assume circular orbits for both the inner and outer orbits, and co-planar triples only since this is what has been observed for low-mass tertiaries \citep[e.g.][]{moe18,tobin18}.  This initial configuration for our assumed formation scenario for Binary 7782, described below, is depicted in Figure~\ref{fig:fig1}.

\begin{figure}[ht!]
\plotone{fig1.eps}
\caption{Cartoon depiction of our proposed scenario for the formation of Binary 7782, specifically mass transfer from an evolved outer tertiary companion on to a compact inner binary via a circumbinary disk.  The outer tertiary component has mass m$_{\rm 3}$, whereas the inner binary components have masses m$_{\rm 1}$ and m$_{\rm 2}$.  The inner and outer orbital separations are denoted by, respectively, a$_{\rm in}$ and a$_{\rm out}$.  The circularization radius of the accretion stream is denoted a$_{\rm c}$, as calculated via Equation~\ref{eqn:ac}, and marks the mean separation of the circumbinary disk.
\label{fig:fig1}}
\end{figure}

We consider a scenario where the outer tertiary companion is filling its Roche lobe and is hence transferring mass to the inner binary.  The mass transfer stream gathers at the circularization radius a$_{\rm c}$, and forms a circumbinary disk.  Using conservation of angular momentum, we can equate the specific angular momentum of the accreted mass at the inner Lagrangian point of the donor star to the final specific angular momentum of the accretion stream at the circularization radius about the inner binary:
\begin{equation}
\label{eqn:specangmom1}
v_{\rm orb,3}(a_{\rm out} - r_{\rm L}) = v_{\rm orb,c}a_{\rm c},
\end{equation}
where R$_{\rm L}$ is the radius of the Roche lobe of the outer tertiary companion, a$_{\rm c}$ is the semi-major axis of the orbit about the inner binary corresponding to the circularization radius and v$_{\rm orb,c}$ is the orbital velocity at a$_{\rm c}$.  The distance from the center of mass corresponding to the outer tertiary companion defined by the Roche lobe is given by \citep{eggleton83}: 
\begin{equation}
\label{eqn:roche}
R_{\rm L} = \frac{0.49q^{2/3}}{0.6q^{2/3} + {\rm ln}(1 + q^{1/3})}a_{\rm out},
\end{equation}
where the mass ratio q is defined as q $=$ m$_{\rm 3}$/(m$_{\rm 1} +$ m$_{\rm 2}$).  Combining Equation~\ref{eqn:roche} with Equation~\ref{eqn:specangmom1}, we can solve for the circularization radius as a function of a$_{\rm out}$ and the assumed stellar masses:
\begin{equation}
\label{eqn:ac}
a_{\rm c} = a_{\rm out}(1 - R_{\rm L}).
\end{equation}
In order for a circumbinary disk to form around the inner binary, we require that a$_{\rm in} <$ a$_{\rm c}$.

Figure~\ref{fig:fig2} shows the parameter space in the P$_{\rm out}$-P$_{\rm in}$-plane for Binary 7782.  We assume initial component masses of m$_{\rm 1} =$ 1.1 M$_{\rm \odot}$ and m$_{\rm 2} =$ 0.9 M$_{\rm \odot}$ for the inner binary components, and m$_{\rm 3} =$ 1.4 M$_{\rm \odot}$ for the outer tertiary.  We compare the circularization radius to the semi-major axis of the inner binary, for which we require a$_{\rm c} >$ a$_{\rm in}$, after folding in all constraints from the requirements for dynamical stability, an outer tertiary that is Roche lobe-filling, and a dynamically hard outer tertiary orbit.  Note that the range of plotted orbital periods P$_{\rm in}$ corresponding to a contact state for the inner binary, assuming R$_{\rm 1} =$ R$_{\rm 2} =$ 1 R$_{\odot}$, lies outside the range of plotted values for P$_{\rm in}$, since it does not contribute significantly to constraining the outer orbital properties of a hypothesized outer WD tertiary.  The thick horizontal solid red line shows the allowed range of outer semi-major axes, after folding in all of the aforementioned criteria.  As is clear, this makes a relatively narrow prediction for the allowed ranges of outer tertiary orbits, namely 2.2 $\times$ 10$^{2}$ days $\le$ P$_{\rm out}$ $\le$ 1.1 $\times$ 10$^3$ days, for our assumed final donor mass.  

\begin{figure}[ht!]
\plotone{fig2.eps}
\caption{Parameter space in the P$_{\rm out}$-P$_{\rm in}$-plane allowed for the hypothetical outer tertiary orbit of Binary 7782.  The solid diagonal black lines show the period corresponding to the circularization radius a$_{\rm c}$ for the mass transfer stream coming from the outer tertiary, for different values of the mass ratio, namely q $=$ 0.1, 0.7, 1, 2 and 10.  We assume initial component masses of m$_{\rm 1}$ = 1.1 M$_{\rm \odot}$ and m$_{\rm 2}$ = 0.9 M$_{\rm \odot}$ for the inner binary components, and m$_{\rm 3}$ is computed for the outer tertiary according to our assumed mass ratio (with our fiducial case corresponding to q $=$ 0.7.  We assume completely conservative mass transfer for this exercise, and a final mass for the outer tertiary of 0.6 M$_{\rm \odot}$ once it has become a WD.  The dashed diagonal black line shows a rough criterion for dynamical stability in the triple, approximately following \citet{mardling99} (i.e., a$_{\rm in} <$ 0.1a$_{\rm out}$ is required for long-term dynamical stability in equal-mass co-planar triples).  The vertical solid red lines show the outer orbital periods corresponding to the hard-soft boundary assuming central velocity dispersions of $\sigma =$ 1, 5 and 10 km s$^{-1}$.  The vertical dashed black lines show the maximum outer orbital period P$_{\rm out}$ for which the outer tertiary companion is Roche lobe-filling, assuming a stellar radius of R$_{\rm 3} =$ 200 R$_{\odot}$ (CHANGE, POSSIBLY USING MESA CALCULATIONS?!).  The horizontal dashed red line shows the observed orbital period for Binary 7782, using its observed orbital period and our assumed final inner companion masses (i.e., m$_{\rm 1} =$ m$_{\rm 2} =$ 1.4 M$_{\odot}$).  Finally, the thick solid horizontal red line shows the parameter space for P$_{\rm out}$ allowed after considering all of the aforementioned criteria.
\label{fig:fig2}}
\end{figure}

\section{Numerical Simulations} \label{sims}

\subsection{AMUSE} \label{amuse}

We use smoothed-particle hydrodynamics (SPH) simulations to compute the time evolution of the mass transfer process.  We adopt the same initial particle masses as in Figure~\ref{fig:fig2}, namely initial component masses of m$_{\rm 1}$ = 1.1 M$_{\rm \odot}$ and m$_{\rm 2}$ = 0.9 M$_{\rm \odot}$ for the inner binary components, and m$_{\rm 3}$ is computed for the outer tertiary according to our assumed mass ratio, with our fiducial case corresponding to q $=$ 0.7.  The initial orbital eccentricities are set to zero.  The inner and outer binary orbital semi-major axes are set to 0.1 AU and 1 AU, respectively (CORRECT).  

\subsection{Initial Conditions} \label{ICs}

In this section, we describe and justify our choice of initial conditions for both our analytic calculations and smoothed-particle hydrodynamics simulations.

We adopt initial component masses of m$_{\rm 1}$ = 1.1 M$_{\rm \odot}$ and m$_{\rm 2}$ = 0.9 M$_{\rm \odot}$ for the inner binary components, and m$_{\rm 3}$ = 1.4 M$_{\rm \odot}$ for the outer tertiary.  This choice for the initial mass of the outer tertiary is critical since it ensures that the donor star during the mass transfer process will have a radiative envelope \citep[e.g.][]{maeder09}.  In turn, this ensures that the mass transfer will be maximally conservative, such that the accretion stream will be maximally stable, accreting at a stable and constant rate \citep[e.g.][]{iben91}.  

\section{Results} \label{results}


\section{Summary and Discussion} \label{discussion}

In this Letter, we have proposed a formation scenario for double BS equal-mass compact binaries, as observed for Binary 7782 in the old open cluster NGC 188.  The proposed scenario involves mass transfer from an evolved outer tertiary companion, which is accreted by the inner binary via a circumbinary disk.  Our scenario makes several predictions for the observed properties of a hypothetical outer triple companion, now a WD.  These are:

\begin{enumerate}

\item For the predicted outer tertiary orbit, the present-day semi-major axis should lie between 220 days $\lesssim$ P$_{\rm out}$ $\lesssim$ 1100 days, assuming initial masses for the inner binary components of m$_{\rm 1} =$ 1.1 M$_{\odot}$ and m$_{\rm 2} =$ 0.9 M$_{\odot}$ and an outer tertiary mass of m$_{\rm 3} =$ 1.4 M$_{\odot}$.

\item Larger final WD masses, and hence core masses for the donor at the time of mass transfer, should correspond to larger final outer orbital periods for the tertiary.  This is because the Roche radius is larger for larger outer orbital periods, such that the donor must evolve to larger radii, and hence core masses, before the onset of mass transfer.

\item For the inner binary, the rotational axes of the BSs should be aligned with each other and the orbital plane of the outer tertiary WD.  This is because accretion onto the BS progenitors proceeds via an accretion disk, that forms at the circularization radius and that has an orbital plane aligned with that of the outer tertiary.

\item The BSs in the inner binary should have roughly equal masses, independent of their initial masses.  This is because it is the lowest mass object that typically accretes the fastest, since its orbital velocity and distance relative to the circumbinary disk is typically the lowest \citep[e.g.][]{haiman09,farris15,rafikov16,kelley17}.  This quickly brings the mass ratio toward unity.  This predicts that the initially lower mass MS star should accrete the most, and should hence be polluted more significantly by any accreted material.  This could be observable in the surface layers of a radiative star.  If the donor is an RGB star, the accretor will be enriched in mostly carbon, oxygen and helium.  If the donor is an AGB star, it will be enriched in mostly s-process elements.

\item Twin BSs in compact binaries formed from the proposed mechanism should be more frequent in younger clusters with ages $\lesssim$ 4-6 Gyr.  This is because clusters with a main-sequence turn-off mass $\lesssim$ 1.2 M$_{\odot}$ have convective envelopes \citep[e.g.][]{iben91,maedoer09}, and a radiative envelope for the donor in a mass transferring binary ensures stable accretion on to the accretor.

WHAT ELSE?  SOMETHING RELATED TO TIMESCALES AND THE PROBABiLitY OF OBSERVING THE WD?

\end{enumerate}


\acknowledgments

N.W.C.L. acknowledges support from a Kalbfleisch Fellowship at the American Museum of Natural History.  

\begin{thebibliography}{}

%\bibitem[\protect\citeauthoryear{Anderson et al.}{2008}]{anderson08}
% Anderson J.,  Sarajedini A., Bedin L. R., King I. R., Piotto G.,
%  Reid I. N., Siegel M., Majewski S. R., Paust N. E. Q., Aparicio A.,
%  Milone A. P., Chaboyer B., Rosenberg A. 2008, AJ, 135, 2055
%\bibitem[\protect\citeauthoryear{Artymowicz et al.}{1993}]{artymowicz93} Artymowicz P., Lin D. N. C., Wampler E. J. 1993, ApJ, 409, 592 
%\bibitem[\protect\citeauthoryear{Bartos et al.}{2017}]{Bartos17} Bartos I., Kocsis B., Haiman Z. \& M\'{a}rka S., 2017, ApJ, 835, 165
%\bibitem[\protect\citeauthoryear{Bartko et al.}{2009}]{Bartko09} Bartko H., Martins F., Trippe S., Fritz T. K., Genzel R., Ott T., Eisenhauer F., Gillessen S., Paumard T., Alexander T., Dodds-Eden K., Gerhard O., Levin Y., Mascetti L., Nayakshin S., Perets H. B., Perrin G., Pfuhl O., Reid M. J., Rouan D., Sternberg A., Trippe S., 2009, ApJ, 697, 1741
%\bibitem[\protect\citeauthoryear{Bartko et al.}{2010}]{Bartko10} Bartko H., Martins F., Fritz T. K., Genzel R., Levin Y., Perets H. B., Paumard T., Nayakshin S., Gerhard O., Alexander T., Dodds-Eden K., Eisenhauer F., Gillessen S., Mascetti L., Ott T., Perrin G., Pfuhl O., Reid M. J., Rouan D., Zilka M., Sternberg A., 2010, ApJ, 708, 834
%\bibitem[\protect\citeauthoryear{Baruteau et al.}{2011}]{Baruteau11} Baruteau C., Cuadra J. \& Lin D.N.C., 2011, ApJ, 726, 28
%\bibitem[\protect\citeauthoryear{Bedin et al.}{2010}]{bedin10} Bedin L. R., Cassisi S., Castelli F., Piotto G., Anderson J., Salaris M., Momany Y., Pietrinferni A. 2010, MNRAS 357, 1038
%\bibitem[\protect\citeauthoryear{Bender et al.}{2005}]{bender05} Bender R., Kormendy J., Bower G., et al. 2005, ApJ, 631, 280
%\bibitem[\protect\citeauthoryear{Bruzual \& Charlot}{2003}]{bruzual03} Bruzual G., Charlot S. 2003, MNRAS, 344, 1000 
%\bibitem[\protect\citeauthoryear{Casassus et al.}{2013}]{casassus13} Casassus S., Hales A., de Gregorio I., Dent B., Belloche A., G\"usten R., M\'enard F., Hughes A. M., Wilner D., Salinas V. 2013, A\&A, 553, 64 
%\bibitem[\protect\citeauthoryear{Ceillier et al.}{2017}]{ceillier17} Ceillier T., Tayar J., Mathur S., Salabert D., Garc\'ia R. A., Stello D., Pinsonneault M. H., van Sanders J., Beck P. G., Bloemen S. 2017, A\& A, 605, 111 
\bibitem[\protect\citeauthoryear{Chatterjee et al.}{2013}]{chatterjee13} Chatterjee S., Rasio F. A., Sills A., Glebbeek E. 2013, ApJ, 777, 106  
\bibitem[\protect\citeauthoryear{Farris et al.}{2015}]{farris15} Farris B. D., Duffell P., MacFayden A. I., Haiman Z. 2014, ApJ, 783, 134 
%\bibitem[\protect\citeauthoryear{Chen \& Amaro-Seoane}{2015}]{chen15} Chen X., Amaro-Seoane P. 2015, Classical and Quantum Gravity, 32, 6 
%\bibitem[\protect\citeauthoryear{Dotter et al.}{2010}]{dotter10} Dotter, A., Sarajedini A., Anderson J., Aparicio A., Bedin L. R., Chaboyer B., Majewski S., Marin-Franch A., Milone A., Paust N., Piotto G., Reid N., Rosenberg A., Siegel M. 2010, ApJ 708, 698

%\bibitem[\protect\citeauthoryear{Dalessandro et al.}{2013}]{dalessandro13} Dalessandro E., Ferraro F. R., Massari D., Lanzoni B., Miocchi R. P., Beccari G., Bellini A., Sills A., Sigurdsson S., Mucciarelli A., Lovisi L. 2013, ApJ, 778, 135
%\bibitem[\protect\citeauthoryear{Eggleton}{1983}]{eggleton83} Eggleton P. P. 1983, ApJ, 268, 368 
%\bibitem[\protect\citeauthoryear{Flock et al.}{2015}]{flock15} Flock M., Ruge J. P., Dzyurkevich N., Henning Th., Klahr H., Wolf S. 2015, A\&A, 574, 68
%\bibitem[\protect\citeauthoryear{Ferraro et al.}{1997}]{ferraro97}
%  Ferraro F. R., Paltrinieri B., Fusi Pecci F., Cacciari C., Dorman
%  B., Rood R. T., Buonanno R., Corsi C. E., Burgarella D., Laget
%  M. 1997, A\&A, 324, 915
%\bibitem[\protect\citeauthoryear{Ferraro et al.}{1999}]{ferraro99}
%  Ferraro F. R., Paltrinieri B., Rood R. T., Dorman B. 1999, ApJ, 522,
%  983
%\bibitem[\protect\citeauthoryear{Ferraro et al.}{2004}]{ferraro04}
%  Ferraro F. R., Beccari G., Rood, R. T., Bellazzini M., Sills A.,
%  Sabbi E. 2004, ApJ, 603, 127
%\bibitem[\protect\citeauthoryear{Ferraro, Valenti \& Origlia}{2006}]{ferraro06} Ferraro F. R., Valenti E., Origlia L. 2006, ApJ, 649, 243
%\bibitem[\protect\citeauthoryear{Ferraro et al.}{2009}]{ferraro09} Ferraro F. R., Beccari G., Dalessandro E., Lanzoni B., Sills A., Rood R. T., Pecci F. F., Karakas A. I., Miocchi P., Bovinelli S. 2009, Nature, 462, 1028  
%\bibitem[\protect\citeauthoryear{Fregeau et al.}{2004}]{fregeau04} Fregeau J. M., Cheung P.,
%  Portegies Zwart S. F., Rasio F. A. 2004, MNRAS, 352, 1
%\bibitem[\protect\citeauthoryear{Fregeau, Ivanova \& Rasio}{2009}]{fregeau09} Fregeau J. M., Ivanova N.,
%  Rasio F. A. 2009, ApJ, 707, 1533
%\bibitem[\protect\citeauthoryear{Geller at al.}{2009}]{geller09}
%  Geller A. M., Mathieu R. D., Harris H. C., McClure R. D., 2009,
%  AJ, 137, 3743
\bibitem[\protect\citeauthoryear{Geller \& Mathieu}{2011}]{geller11}
  Geller A. M., Mathieu R. D. 2011, Nature, 478, 356
\bibitem[\protect\citeauthoryear{Geller \& Mathieu}{2012}]{geller12} Geller A. M., Mathieu R. D. 2012, AJ, 144, 54
\bibitem[\protect\citeauthoryear{Geller et al.}{2013a}]{geller13a} Geller A. M.,
de Grijs R., Li C., Hurley J. R. 2013, ApJ, 779, 30
\bibitem[\protect\citeauthoryear{Geller et al.}{2013b}]{geller13b} Geller A. M., Hurley J. R., Mathieu R. D. 2013, AJ, 145, 8
\bibitem[\protect\citeauthoryear{Geller \& Leigh}{2015}]{geller15} Geller A. M., Leigh W. W. C. 2015, ApJL, 808, L25 
%\bibitem[\protect\citeauthoryear{Ghez}{2008}]{ghez08} Ghez A. M., Salim S., Weinberg N. N., Lu J. R., Do T., Dunn J. K., Matthews K, Morris M. R., Yelda S., Becklin E. E., Kremenek T., Milosavlevic M., Naiman J., ApJ, 2008, 689, 1044 
%\bibitem[\protect\citeauthoryear{Ghez}{2008}]{ghez08}
\bibitem[\protect\citeauthoryear{Gosnell et al.}{2014}]{gosnell14} Gosnell N. M., Mathieu R. D., Geller A. M., Sills A., Leigh N. W. C., Knigge C. 2014, ApJ, 783, 8
\bibitem[\protect\citeauthoryear{Gosnell et al.}{2015}]{gosnell15} Gosnell N. M., Mathieu R. D., Geller A. M., Sills A., Leigh N. W. C., Knigge C. 2015, ApJ, 814, 163
\bibitem[\protect\citeauthoryear{Haiman et al.}{2009}]{haiman09} Haiman Z., Kocsis B., Menou K. 2009, ApJ, 700, 1952  
%\bibitem[\protect\citeauthoryear{Graham \& Spitler}{2009}]{Graham09} Graham A.W. \& Spitler L.R., 2009, MNRAS, 397, 2148
%\bibitem[\protect\citeauthoryear{Harris et al.}{1996; 2010 update}]{harris96} Harris W. E. 1996, AJ, 112,
%  1487; 2010 update
%\bibitem[\protect\citeauthoryear{Heggie}{1975}]{heggie75} Heggie D. C. 1975, MNRAS, 173, 729
%\bibitem[\protect\citeauthoryear{Heggie \& Hut}{2003}]{heggie03}
%  Heggie D. C., Hut P. 2003, The Gravitational Million-Body Problem:
%  A Multidisciplinary Approach to Sar Cluster Dynamics (Cambridge:
%  Cambridge University Press)
\bibitem[\protect\citeauthoryear{Hills}{1975}]{hills75} Hills J. G. 1975, AJ, 80, 809
%\bibitem[\protect\citeauthoryear{Hurley et al.}{2005}]{hurley05}
%  Hurley J. R., Pols O. R., Aarseth S. J., Tout C. A. 2005, MNRAS,
%  363, 293
%\bibitem[\protect\citeauthoryear{Hurley, Aarseth \&
%    Shara}{2007}]{hurley07} Hurley J. R., Aarseth S. J., Shara M. M. 2007, ApJ, 665, 707
%\bibitem[\protect\citeauthoryear{Hut \& Bahcall}{1983}]{hut83} Hut P., Bahcall J. N. 1983, ApJ,
%  268, 319
%\bibitem[\protect\citeauthoryear{Howarth}{1983}]{howarth83} Howarth I. D. 1983, MNRAS, 203, 301
\bibitem[\protect\citeauthoryear{Hypki \& Giersz}{2013}]{hypki13} Hypki A., Giersz M. 2013,
MNRAS, 429, 1221
\bibitem[\protect\citeauthoryear{Iben}{1991}]{iben91}  Iben I., Jr. 1991, ApJS, 76, 55
\bibitem[\protect\citeauthoryear{Kelley et al.}{2017}]{kelley17} Kelley L. Z., Blecha L., Hernquist L. 2017, MNRAS, 464, 3131 
\bibitem[\protect\citeauthoryear{Knigge, Leigh \& Sills}{2009}]{knigge09} Knigge C., Leigh
  N., Sills A. 2009, Nature, 457, 288
%\bibitem[\protect\citeauthoryear{Laidler et al.}{2008}]{laidler08} Laidler et al. 2008, ``Synphot Data User's Guide'' (Baltimore, STScI)
%\bibitem[\protect\citeauthoryear{Lanzoni et al.}{2007}]{lanzoni07} Lanzoni B., Dalessandro E., Ferraro F. R., Valenti E., Beccari G., Schiavon R. P., Rood R. T., Mapelli M., Sigurdsson S. 2007, ApJ, 663, 1040
%\bibitem[\protect\citeauthoryear{Kaluzny et al.}{1998}]{kaluzny98} Kaluzny J., Hilditch R. W., Clement C., Rucinski S. M. 1998, MNRAS, 296, 347
%\bibitem[\protect\citeauthoryear{Kaluzny et al.}{2015}]{kaluzny15} Kaluzny J., Thompson I. B., Narloch W., Pych W., Rozyczka M. 2015, Acta Astronomica, 65, 267 
\bibitem[\protect\citeauthoryear{Leigh, Sills \& Knigge}{2007}]{leigh07} Leigh
  N. W. C., Sills A., Knigge C. 2007, ApJ, 661, 210
%\bibitem[\protect\citeauthoryear{Leigh, Sills \& Knigge}{2008}]{leigh08} Leigh
%  N. W., Sills A., Knigge C. 2008, ApJ, 678, 564
%\bibitem[\protect\citeauthoryear{Leigh, Sills \& Knigge}{2009}]{leigh09} Leigh
%  N. W., Sills A., Knigge C. 2009, MNRAS, 399, L179
\bibitem[\protect\citeauthoryear{Leigh \& Sills}{2011}]{leigh11} Leigh N. W. C., Sills A. 2011, MNRAS, 410, 2370
\bibitem[\protect\citeauthoryear{Leigh et al.}{2012}]{leigh12} Leigh N. W. C., Umbreit S.,
Sills A., Knigge C., De Marchi G., Glebbeek E., Sarajedini A. 2012, MNRAS, 422, 1592
\bibitem[\protect\citeauthoryear{Leigh et al.}{2013}]{leigh13} Leigh N. W. C., Knigge C.,
Sills A., Perets H. B., Sarajedini A., Glebbeek E. 2013, MNRAS, 428, 897
%\bibitem[\protect\citeauthoryear{Leigh et al.}{2014}]{leigh14} Leigh N. W. C., L\"utzgendorf N., 
%Geller A. M., Maccarone T. J., Heinke C., Sesana A. 2014, MNRAS, 444, 29
%\bibitem[\protect\citeauthoryear{Leigh et al.}{2016a}]{leigh16a} Leigh N. W. C., Antonini F., Stone N. C., Shara M. M., Merritt D. 2016a, MNRAS, 463, 1605
%\bibitem[\protect\citeauthoryear{Leigh et al.}{2016b}]{leigh16b} Leigh N. W. C., Stone N. C., Geller A. M., Shara M. M., Muddu H., Solano-Oropeza D., Thomas Y. 2016b, MNRAS, 463, 3311 
%\bibitem[\protect\citeauthoryear{Leigh et al.}{2016c}]{leigh16c} Leigh N. W. C., Geller A. M., Toonen S. 2016c, ApJ, 818, 21
%\bibitem[\protect\citeauthoryear{Leigh et al.}{2016d}]{leigh16d} Leigh N. W. C., Shara M. M., Geller A. M. 2016d, MNRAS, 459, 1242
\bibitem[\protect\citeauthoryear{Leonard}{1989}]{leonard89} Leonard
  P. J. T. 1989, AJ, 98, 217
%\bibitem[\protect\citeauthoryear{Leonard \& Linnell}{1992}]{leonard92}
%  Leonard P. J. T., Linnell A. P. 1992, AJ, 103, 1928
%\bibitem[\protect\citeauthoryear{Leonard \& Livio}{1995}]{leonard95} Leonard P. J. T., Livio M. 1995,
%  ApJ, 447, L121
%\bibitem[\protect\citeauthoryear{Lim, Diaz \& Laidler}{2015}]{lim15} Lim P. L., Diaz R. L., Laidler V. 2015, PySynphot User's Guide (Baltimore, MD: STScI)
%\bibitem[\protect\citeauthoryear{Lovisi et al.}{2012}]{lovisi12} Lovisi L., Mucciarelli A., Lanzoni B., Ferraro F. R., Gratton R., Dalessandro E., Contreras R. R. 2012, ApJ, 754, 91
\bibitem[\protect\citeauthoryear{Maeder}{2009}]{maeder09} Maeder A. 2009, 
Physics, Formation and Evolution of Rotating Stars. Berlin: Springer-Verlag
\bibitem[\protect\citeauthoryear{Mardling \& Aarseth}{1999}]{mardling99} Mardling R. A., Aarseth S. J. 1999, ASIC, 522, 385  
%\bibitem[\protect\citeauthoryear{Mapelli et al.}{2006}]{mapelli06}
%  Mapelli M., Sigurdsson S., Ferraro F. R., Colpi M., Possenti A., Lanzoni B. 2006,
%  MNRAS, 373, 361
\bibitem[\protect\citeauthoryear{Mathieu \& Geller}{2009}]{mathieu09}
  Mathieu R. D., Geller A. R. 2009, Nature, 462, 1032
\bibitem[\protect\citeauthoryear{McCrea}{1964}]{mccrea64} McCrea
  W. H. 1964, MNRAS, 128, 147
\bibitem[\protect\citeauthoryear{Moe \& Kratter}{2018}]{moe18} Moe M., Kratter K. M. 2018, ApJ, 854, 44 
%\bibitem[\protect\citeauthoryear{Milone et al.}{2008}]{milone08}
%  Milone A. P., Piotto G., Bedin L. R., Sarajedini A. 2008, MmSAI, 79,
%  623
%\bibitem[\protect\citeauthoryear{Milone et al.}{2012}]{milone12} Milone A. P., Piotto G.,
%Bedin L. R., Aparicio A., Anderson J., Sarajedini A., Marino A. F., Moretti A.,
%Davies M. B., Chaboyer B., Dotter A., Hempel M., Marin-Franch A., Majewski S.,
%Paust N. E. Q., Reid I. N., Rosenberg A., Siegel M. 2012, A\&A, 540, 16
\bibitem[\protect\citeauthoryear{Perets \& Fabrycky}{2009}]{perets09} Perets H. B., Fabrycky D. C. 2009, ApJ, 697, 1048
%\bibitem[\protect\citeauthoryear{Pickles}{1998}]{pickles98} Pickles A. J. 1998, PASP, 110, 863
\bibitem[\protect\citeauthoryear{Piotto et al.}{2004}]{piotto04}
  Piotto G., De Angeli F., King I. R., Djorgovski S. G., Bono G.,
  Cassisi S., Meylan G., Recio-Blanco A., Rich R. M., Davies M. B. 2004,
  ApJ, 604, L109
\bibitem[\protect\citeauthoryear{Rafikov}{2016}]{rafikov16} Rafikov R. R. 2016, ApJ, 827, 111 
%\bibitem[\protect\citeauthoryear{Piotto et al.}{2007}]{piotto07}
%Piotto G., Bedin L. R., Anderson J., King I. R., Cassisi S., Milone A. P., Villanova S., Pietrinferni A., Renzini A. 2007, ApJ, 661, L53
%\bibitem[\protect\citeauthoryear{Piotto et al.}{2015}]{piotto15}
%  Piotto G., Milone A. P., Bedin L. R., Anderson J., King I. R., Marino A. F., Nardiello D., Aparicio A., Barbuy B., Bellini A., Brown T. M., Cassisi S., Cool A. M., Cunial A., Dalessandro E., D'Antona F., Ferraro F. R., Hidalgo S., Lanzoni B., Monelli M., Ortolani S., Renzini A., Sarajedini A., van der Marel R. P., Vesperini E., Zoccali M. 2015, AJ, 149, 91
%\bibitem[\protect\citeauthoryear{Raghavan et al.}{2010}]{raghavan10} Raghavan D., McAlister H. A., Henry T. J, et al. 2010, ApJS, 190, 1 
\bibitem[\protect\citeauthoryear{Sandquist et al.}{2003}]{sandquist03} Sandquist E. L., Latham D. W., Shetrone M. D., Milone A. A. E. 2003, AJ, 125, 810
%\bibitem[\protect\citeauthoryear{Sarajedini et al.}{2007}]{sarajedini07}
%  Sarajedini A., Bedin L. R., Chaboyer B., Dotter  A., Siegel M.,
%  Anderson J., Aparicio A., King I., Majewski S., Marin-Franch A.,
%  Piotto G., Reid  I. N., Rosenberg A., Steven M. 2007, AJ, 133, 1658
%\bibitem[\protect\citeauthoryear{Sigurdsson \& Phinney}{1993}]{sigurdsson93} Sigurdsson S. \& Phinney S.L. 1993, ApJ, 415, 631 
%\bibitem[\protect\citeauthoryear{Shara et al.}{1995}]{shara95} Shara
%  M. M., Drissen L., Bergeron L. E.,  Paresce F. 1995, ApJ, 441, 617
\bibitem[\protect\citeauthoryear{Shara, Saffer \&
    Livio}{1997}]{shara97} Shara M. M., Saffer R. A., Livio M. 1997,
  ApJ, 489, L59
\bibitem[\protect\citeauthoryear{Sills et al.}{1997}]{sills97} Sills A., Lombardi J. C. Jr.,
Bailyn C. D., Demarque P., Rasio F. A., Shapiro S. L. 1997, ApJ, 487, 290
\bibitem[\protect\citeauthoryear{Sills \& Bailyn}{1999}]{sills99} Sills A., Bailyn C. D. 1999,
  ApJ, 513, 428
\bibitem[\protect\citeauthoryear{Sills et al.}{2001}]{sills01} Sills A. R., Faber J. A.,
  Lombardi J. C., Rasio F. A., Waren A. R. 2001, ApJ, 548, 323
\bibitem[\protect\citeauthoryear{Simunovic, Puzia \& Sills}{2014}]{simunovic14} Simunovic, M., Puzia, T. H., Sills, A. 2014, ApJL, 795, L10
\bibitem[\protect\citeauthoryear{Simunovic \& Puzia}{2016}]{simunovic16} Simunovic M., Puzia T. H. 2016, MNRAS 462, 3401
\bibitem[\protect\citeauthoryear{Tobin et al.}{2018}]{tobin18} Tobin J. J., Looney L. W., Li Z.-Y., Sadavoy S. I., Dunham M. M. Segura-Cox D., Kratter K., Chandler C. J., Melis C., Harris R. J., Perez L. 2018, ApJ, 867, 43 
%\bibitem[\protect\citeauthoryear{Sirianni et al.}{2005}]{sirianni05} Sirianni M., Jee M. J., Benitez N., Blakeslee J. P., Martel A. R., Meurer G., Clampin M., De Marchi G., Ford H. C., Gilliland R., Hartig G. F., Illingworth G. D., Mack J., McCann W. J. 2005, PASP, 117, 1049
%\bibitem[\protect\citeauthoryear{Sollima et al.}{2007}]{sollima07}
%  Sollima A., Beccari G., Ferraro F. R., Fusi Pecci F., Sarajedini
%  A. 2008, MNRAS, 380, 781
%\bibitem[\protect\citeauthoryear{Sollima et al.}{2008}]{sollima08}
%  Sollima A., Beccari G., Ferraro F. R., Fusi Pecci F., Sarajedini
%  A. 2007, A\&A, 481, 701
%\bibitem[\protect\citeauthoryear{Soto et al.}{2017}]{soto17} Soto M., Belloni A., Anderson J., Piotto G., Bedin L. R., van der Marel R. P., Milone A. P., Brown T. M., Cool A. M., King I. R., Sarajedini A., Granata V., Cassisi S., Aparicio A., Hidalgo S., Ortolani S., Nardiello D. 2017, AJ, 153, 19
%\bibitem[\protect\citeauthoryear{Spitzer}{1969}]{spitzer69} Spitzer
%  L. Jr. 1969, ApJ, 168, L139
%\bibitem[\protect\citeauthoryear{Spitzer}{1987}]{spitzer87} Spitzer L. 1987, Dynamical Evolution of Globular Clusters (Princeton: Princeton University Press)
%\bibitem[\protect\citeauthoryear{Tout et al.}{1996}]{tout96} Tout C. A., Pols O. R., Eggleton P. P., Han Z. 1996, MNRAS, 281, 257
\bibitem[\protect\citeauthoryear{van den Berg et al.}{2001}]{vandenberg01} van den Berg M., Orosz J., Verbunt F., Stassun K. 2001, A\&A, 375, 375

\end{thebibliography}

\end{document}
