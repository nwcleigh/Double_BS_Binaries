%%
%% Beginning of file 'sample62.tex'
%%
%% Modified 2018 January
%%
%% This is a sample manuscript marked up using the
%% AASTeX v6.2 LaTeX 2e macros.
%%
%% AASTeX is now based on Alexey Vikhlinin's emulateapj.cls 
%% (Copyright 2000-2015).  See the classfile for details.

%% AASTeX requires revtex4-1.cls (http://publish.aps.org/revtex4/) and
%% other external packages (latexsym, graphicx, amssymb, longtable, and epsf).
%% All of these external packages should already be present in the modern TeX 
%% distributions.  If not they can also be obtained at www.ctan.org.

%% The first piece of markup in an AASTeX v6.x document is the \documentclass
%% command. LaTeX will ignore any data that comes before this command. The 
%% documentclass can take an optional argument to modify the output style.
%% The command below calls the preprint style  which will produce a tightly 
%% typeset, one-column, single-spaced document.  It is the default and thus
%% does not need to be explicitly stated.
%%
%%
%% using aastex version 6.2
%%\documentclass{aastex62}
\documentclass[twocolumn]{aastex62}


%% The default is a single spaced, 10 point font, single spaced article.
%% There are 5 other style options available via an optional argument. They
%% can be envoked like this:
%%
%% \documentclass[argument]{aastex62}
%% 
%% where the layout options are:
%%
%%  twocolumn   : two text columns, 10 point font, single spaced article.
%%                This is the most compact and represent the final published
%%                derived PDF copy of the accepted manuscript from the publisher
%%  manuscript  : one text column, 12 point font, double spaced article.
%%  preprint    : one text column, 12 point font, single spaced article.  
%%  preprint2   : two text columns, 12 point font, single spaced article.
%%  modern      : a stylish, single text column, 12 point font, article with
%% 		  wider left and right margins. This uses the Daniel
%% 		  Foreman-Mackey and David Hogg design.
%%  RNAAS       : Preferred style for Research Notes which are by design 
%%                lacking an abstract and brief. DO NOT use \begin{abstract}
%%                and \end{abstract} with this style.
%%
%% Note that you can submit to the AAS Journals in any of these 6 styles.
%%
%% There are other optional arguments one can envoke to allow other stylistic
%% actions. The available options are:
%%
%%  astrosymb    : Loads Astrosymb font and define \astrocommands. 
%%  tighten      : Makes baselineskip slightly smaller, only works with 
%%                 the twocolumn substyle.
%%  times        : uses times font instead of the default
%%  linenumbers  : turn on lineno package.
%%  trackchanges : required to see the revision mark up and print its output
%%  longauthor   : Do not use the more compressed footnote style (default) for 
%%                 the author/collaboration/affiliations. Instead print all
%%                 affiliation information after each name. Creates a much
%%                 long author list but may be desirable for short author papers
%%
%% these can be used in any combination, e.g.
%%
%% \documentclass[twocolumn,linenumbers,trackchanges]{aastex62}
%%
%% AASTeX v6.* now includes \hyperref support. While we have built in specific
%% defaults into the classfile you can manually override them with the
%% \hypersetup command. For example,
%%
%%\hypersetup{linkcolor=red,citecolor=green,filecolor=cyan,urlcolor=magenta}
%%
%% will change the color of the internal links to red, the links to the
%% bibliography to green, the file links to cyan, and the external links to
%% magenta. Additional information on \hyperref options can be found here:
%% https://www.tug.org/applications/hyperref/manual.html#x1-40003
%%
%% If you want to create your own macros, you can do so
%% using \newcommand. Your macros should appear before
%% the \begin{document} command.
%%
\newcommand{\vdag}{(v)^\dagger}
\newcommand\aastex{AAS\TeX}
\newcommand\latex{La\TeX}

\newcommand{\MSun}{\mbox{M$_\odot$}}
\newcommand{\RSun}{\mbox{R$_\odot$}}
\newcommand{\LSun}{\mbox{L$_\odot$}}

\def\apgt{\ {\raise-.5ex\hbox{$\buildrel>\over\sim$}}\ }
\def\aplt{\ {\raise-.5ex\hbox{$\buildrel<\over\sim$}}\ }
%%\def\figure{\ {figure\,}}

\def\Simon#1{{\bf {\color{red}[#1 -- Simon]}}}
\def\simon#1{{\bf {\color{red}[#1 -- Simon]}}}
\def\del#1{{\bf {\sout{#1}}}}
\def\replace#1#2{{\bf {\sout{#1} $\rightarrow$ {\bf #2}}}}


%% Reintroduced the \received and \accepted commands from AASTeX v5.2
\received{January 1, 2018}
\revised{January 7, 2018}
\accepted{\today}
%% Command to document which AAS Journal the manuscript was submitted to.
%% Adds "Submitted to " the arguement.
\submitjournal{ApJ}



%% Mark up commands to limit the number of authors on the front page.
%% Note that in AASTeX v6.2 a \collaboration call (see below) counts as
%% an author in this case.
%
%\AuthorCollaborationLimit=3
%
%% Will only show Schwarz, Muench and "the AAS Journals Data Scientist 
%% collaboration" on the front page of this example manuscript.
%%
%% Note that all of the author will be shown in the published article.
%% This feature is meant to be used prior to acceptance to make the
%% front end of a long author article more manageable. Please do not use
%% this functionality for manuscripts with less than 20 authors. Conversely,
%% please do use this when the number of authors exceeds 40.
%%
%% Use \allauthors at the manuscript end to show the full author list.
%% This command should only be used with \AuthorCollaborationLimit is used.

%% The following command can be used to set the latex table counters.  It
%% is needed in this document because it uses a mix of latex tabular and
%% AASTeX deluxetables.  In general it should not be needed.
%\setcounter{table}{1}

%%%%%%%%%%%%%%%%%%%%%%%%%%%%%%%%%%%%%%%%%%%%%%%%%%%%%%%%%%%%%%%%%%%%%%%%%%%%%%%%
%%
%% The following section outlines numerous optional output that
%% can be displayed in the front matter or as running meta-data.
%%
%% If you wish, you may supply running head information, although
%% this information may be modified by the editorial offices.
\shorttitle{Compact Binaries of BS Twins from Stellar Triples}
\shortauthors{Portegies Zwart \& Leigh}
%%
%% You can add a light gray and diagonal water-mark to the first page 
%% with this command:
% \watermark{text}
%% where "text", e.g. DRAFT, is the text to appear.  If the text is 
%% long you can control the water-mark size with:
%  \setwatermarkfontsize{dimension}
%% where dimension is any recognized LaTeX dimension, e.g. pt, in, etc.
%%
%%%%%%%%%%%%%%%%%%%%%%%%%%%%%%%%%%%%%%%%%%%%%%%%%%%%%%%%%%%%%%%%%%%%%%%%%%%%%%%%

%% This is the end of the preamble.  Indicate the beginning of the
%% manuscript itself with \begin{document}.

\begin{document}

\title{A Triple Origin for Twin Blue Stragglers in Close Binaries}

%% LaTeX will automatically break titles if they run longer than
%% one line. However, you may use \\ to force a line break if
%% you desire. In v6.2 you can include a footnote in the title.

%% A significant change from earlier AASTEX versions is in the structure for 
%% calling author and affilations. The change was necessary to implement 
%% autoindexing of affilations which prior was a manual process that could 
%% easily be tedious in large author manuscripts.
%%
%% The \author command is the same as before except it now takes an optional
%% arguement which is the 16 digit ORCID. The syntax is:
%% \author[xxxx-xxxx-xxxx-xxxx]{Author Name}
%%
%% This will hyperlink the author name to the author's ORCID page. Note that
%% during compilation, LaTeX will do some limited checking of the format of
%% the ID to make sure it is valid.
%%
%% Use \affiliation for affiliation information. The old \affil is now aliased
%% to \affiliation. AASTeX v6.2 will automatically index these in the header.
%% When a duplicate is found its index will be the same as its previous entry.
%%
%% Note that \altaffilmark and \altaffiltext have been removed and thus 
%% can not be used to document secondary affiliations. If they are used latex
%% will issue a specific error message and quit. Please use multiple 
%% \affiliation calls for to document more than one affiliation.
%%
%% The new \altaffiliation can be used to indicate some secondary information
%% such as fellowships. This command produces a non-numeric footnote that is
%% set away from the numeric \affiliation footnotes.  NOTE that if an
%% \altaffiliation command is used it must come BEFORE the \affiliation call,
%% right after the \author command, in order to place the footnotes in
%% the proper location.
%%
%% Use \email to set provide email addresses. Each \email will appear on its
%% own line so you can put multiple email address in one \email call. A new
%% \correspondingauthor command is available in V6.2 to identify the
%% corresponding author of the manuscript. It is the author's responsibility
%% to make sure this name is also in the author list.
%%
%% While authors can be grouped inside the same \author and \affiliation
%% commands it is better to have a single author for each. This allows for
%% one to exploit all the new benefits and should make book-keeping easier.
%%
%% If done correctly the peer review system will be able to
%% automatically put the author and affiliation information from the manuscript
%% and save the corresponding author the trouble of entering it by hand.

\correspondingauthor{Nathan W. C. Leigh}
\email{nleigh@amnh.org}

\author{Simon Portegies Zwart}
\affiliation{Leiden Observatory \\
Leiden University \\
PO Box 9513, 2300 RA \\
Leiden, the Netherlands}

\author{Nathan W. C. Leigh}
\affil{American Museum of Natural History \\
Department of Astrophysics \\
79th Street at Central Park West \\
New York, NY 10024-5192, USA}
\affil{Stony Brook University \\
Department of Physics and Astronomy\\
Stony Brook, NY 11794-3800, USA}
\affil{Departamento de Astronom\'ia \\ 
Facultad de Ciencias F\'isicas y Matem\'aticas \\ 
Universidad de Concepci\'on \\ 
Concepci\'on, Chile}


%\author{Simon Portegies Zwart}
%\affiliation{Leiden Observatory \\
%Leiden University \\
%PO Box 9513, 2300 RA \\
%Leiden, the Netherlands}


%% Note that the \and command from previous versions of AASTeX is now
%% depreciated in this version as it is no longer necessary. AASTeX 
%% automatically takes care of all commas and "and"s between authors names.

%% AASTeX 6.2 has the new \collaboration and \nocollaboration commands to
%% provide the collaboration status of a group of authors. These commands 
%% can be used either before or after the list of corresponding authors. The
%% argument for \collaboration is the collaboration identifier. Authors are
%% encouraged to surround collaboration identifiers with ()s. The 
%% \nocollaboration command takes no argument and exists to indicate that
%% the nearby authors are not part of surrounding collaborations.

%\newcommand{\mnras}{\textrm{MNRAS}}
%\newcommand{\apj}{\textrm{ApJ}}
%\newcommand{\aap}{\textrm{A\&A}}
%\newcommand{\apjl}{\textrm{ApJ}}

\begin{abstract}

We propose that twin blue stragglers (BSs) in compact binaries evolve
through a phase of mass transfer from a giant outer tertiary companion
on to the inner binary.  We apply this scenario to the twin-BS binary
WOCS~ID~7782 in the old open cluster NGC~188.  This binary has two
comparable-mass main-sequence stars in an almost circular ($e \aplt
0.1$) orbit of $\aplt 10$\,days.  Our theoretical arguments are
supported by simulations of an inner binary that accretes from a outer
Roche-lobe overfilling star using the Astrophysicsal Multipurpose
Software Environment. At least 80\,\% of the mass liberated by the
tertiary star goes through a circum-binary disk before it is accreted
by in the inner binary, causing the two stars to turn into blue
stragglers.  In order to acquire relatively stable phase of mass
transfer the donor should be about 1.4\,\MSun\, and over-fill it's
Roche lobe before ascending the asymptotic giant-branch.  The outer
star eventually turns into a 0.43 to 0.54\,\MSun\, white dwarf in a
relatively wide $\apgt 5.8\,yr$ orbit.  Although, the scenario is
generic, it requires some fine-tuning to achieve parameters comparable
to the observed twin WOCS~ID~7782.  This system is best reproduced
when starting with a $m_{\rm 3} = 1.4$ M$_{\odot}$ outer star in an
220 to 1100~day orbit around an inner binary composed of an
1.1\,\,MSun\, primary and a $m_{\rm 2}=0.7$ to $0.9$M$_{\odot}$
secondary star in an 8.6 to 24\,day orbit.  Based on our simulations
and theoretical arguments we predict that twin blue-stragglers that
formed through mass transfer from a Roche-lobe over-filling outer star
are generally comparable in mass, have aligned spins which again are
algined with the orbit of the tertiary white-dwarf star.  If the two
inner stars ware initially unequal in mass the less massive star will
have accreted more material from the tertiary star, which therefore is
more enhanced in CNO-processed material.

\end{abstract}

\keywords{stars: blue stragglers -- binaries: general -- globular clusters: general -- scattering}


%%%%%%%%%%%%%%%%%%%%%%%%%%%%%%%
\section{Introduction} \label{intro}

Most blue straggler stars are brighter and bluer than the main-sequence
(MS) turn-off in a cluster colour-magnitude diagram
\citep[e.g.][]{1953AJ.....58...61S,1989AJ.....98..217L,2014ApJ...782...49S}.
Two primary channels for BS formation have been proposed: mass
transfer from an evolved donor on to a MS star in a binary star system
\citep[e.g.][]{1964MNRAS.128..147M,1997A&A...328..143P,2009Natur.457..288K,2011MNRAS.410.2370L,2011Natur.478..356G},
and direct stellar collisions involving MS stars likely mediated via
binaries
\citep[e.g.][]{1975AJ.....80..809H,1997A&A...328..130P,2007ApJ...661..210L,2013MNRAS.428..897L,2013MNRAS.429.1221H, 2019A&A...621L..10P}.
The first mechanism predicts BSs in binaries with WD companions,
whereas the second predicts MS companions in a wide and eccentric
binary.  Other possible, albeit related, formation mechanisms include
mergers of close MS-MS binaries \citep{2019A&A...621L..10P}, and
mergers of the inner binaries of hierarchical triple star systems
induced by Lidov-Kozai oscillations coupled with tidal damping
\citep[e.g.][]{2009ApJ...697.1048P}.  The latter predicts no binary companion, whereas the former predicts a MS companion in a wide binary.

In spite of these specific predictions for the expected properties of
BSs formed from each of the above production mechanisms, many BSs
exist with observed properties that defy these simple scenarios.  For
example, in the old open cluster (OC) M67, there lurks a candidate triple
system that is posited to host two BSs
\citep{2001A&A...375..375V,2003AJ....125..810S}; one in the inner binary and one as the outer triple companion \citep{2003AJ....125..810S,2001A&A...375..375V}.  
%The observations
%suggest that the outer tertiary is itself a BS, with a mass $\sim$ 1.7
%M$_{\odot}$ and orbiting the inner binary with a period of $\sim
%1188.5$ days \citep{2003AJ....125..810S}.  The inner binary has a
%period of only $\sim 1.068$ days \citep{2001A&A...375..375V}, and
%hosts a BS of mass $\sim 2.52$ M$_{\odot}$.  
In order to reproduce the total system mass 
at least five stars are needed \citep{2011MNRAS.410.2370L}, which is strongly indicative of a
dynamical origin for the system; a single direct interaction
involving a binary and a triple that resulted in two separate
collisions is the most probable explanation for its origin
(instead of back-to-back direct binary-binary interactions)
\citep{2004MNRAS.350..615G,2011MNRAS.410.2370L}.  

Even more curious, there exists in the old OC NGC 188 a double BS
binary, called WOCS 7782 \citep{2009AJ....137.3743G}.  The BS
population in NGC 188 has a bi-modal period-eccentricity distribution.
As discussed in \citet{2011MNRAS.410.2370L}, this could be hinting at
a triple origin for at least some subset of the total BS population.
As for WOCS 7782, \citet{2009Natur.462.1032M} observed a compact and
mildly eccentric (i.e., $e \sim 0.1$) binary star system with an
orbital period of $\sim$ 10 days hosting two roughly equal-mass BSs.
During a given binary-binary interaction, the probability that not one
but two direct (MS-MS) collisions will occur is less than $10^{-2}$
\citep{1989AJ.....98..217L,2011MNRAS.410.2370L,2012MNRAS.425.2369L}.
Plus, binaries with collision products typically have relatively long
orbital periods \citep{2011Sci...334.1380F}. Dynamically, it is
difficult to form a short-period binary composed of two collision
products during a collisional interaction in a star cluster
\citep{2011MNRAS.410.2370L,2011Sci...334.1380F}, and the timescale for
exchanging another BS into a pre-existing BS-MS or BS-WD binary is
much longer than the expected BS lifetime (see
\citet{2011MNRAS.410.2370L} and the end of Section~\ref{sect:dyn}
below).  So, how did WOCS 7782 form?

We propose a formation channel for WOCS 7782, and compact double BS
binaries in general, which involves mass transfer from an outer
tertiary companion on to an inner binary composed of two MS stars.  In
section~\ref{sect:dyn}, we constrain the range of initial (i.e.,
pre-mass transfer) orbital parameters for a hypothetical outer
tertiary companion, using a combination of dynamical and stellar
evolution-based constraints.  In Section~\ref{sims} we present the
numerical simulations used to study the mass transfer process in this
triple system. We adopt orbital parameters that, according to our
expectations, are most promising for the progenitors of the twin BS
7782.  The calculations are performed using the Astrophysical
Multipurpose Software Environment \cite[\texttt{AMUSE} for short,
  see][]{PortegiesZwart2013456,AMUSE} with a combination of stellar
evolution, hydrodynamical and gravitational simulations.  With these
calculations we further constrain the possible range of initial
parameters that naturally lead to twin BSs with orbital parameters
similar to the 7782 system, without exhaustively covering parameter
space.  We summarize and discuss the implications of our results for
compact double BS binaries and, more generally, mass transfer in
stellar triples in Section~\ref{sect:discussion}.

\section{Constraints on the present-day orbital parameters for a hypothesized
         tertiary companion in the compact BS binary WOCS 7782} \label{sect:dyn}

In our scenario, we start with a binary star with component masses
$m_{\rm 1}$ and $m_{\rm 2}$ that is orbited by a tertiary of mass
$m_3$. The inner and outer binary orbital semi-major axes are denoted
a$_{\rm in}$ and a$_{\rm out}$, respectively.  For specificty, we
assume both orbits, the inner as well as the outer, to be circular and
in the same plane, which minimizes chaotic effects during the mass
transfer process, facilitates more stable mass transfer and ultimately
allows us to simulate our target system for longer while also
maximizing the amount of mass transferred.  These assumptions are also
supported by the population of observed low-mass triples
\citep{2010yCat..73890925T,2018ApJ...854...44M}.  This initial
configuration for our assumed formation scenario for WOCS 7782,
described below, is depicted in figure~\ref{fig:fig1}.

\begin{figure}[ht!]
%\plotone{fig_doodle.eps}
\includegraphics[width=\columnwidth]{fig_doodle.pdf}
\caption{Cartoon depiction of our proposed scenario for the formation
  of WOCS 7782, specifically mass transfer from an evolved outer
  tertiary companion on to a compact inner binary via a circumbinary
  disk.  The outer tertiary component has mass m$_{\rm 3}$, whereas
  the inner binary components have masses m$_{\rm 1}$ and m$_{\rm 2}$.
  The inner and outer orbital separations are denoted by,
  respectively, a$_{\rm in}$ and a$_{\rm out}$.  The circularization
  radius of the accretion stream is denoted a$_{\rm c}$, as calculated
  via Equation~\ref{eqn:ac}, and marks the mean separation of the
  circumbinary disk.
\label{fig:fig1}}
\end{figure}

According to our scenario $m_3 > m_1 > m_2$ and the outer orbit is
sufficiently small that the tertiary star is filling its Roche lobe
and transfers mass to the inner binary before it reaches the asymptotic
giant branch. We constrain the inner orbit by requiring the triple
system to be dynamically stable, for which we adopt eq.\,1 in
\cite{1999ASIC..522..385M}.  While transferring mass, the accretion
stream gathers around the inner binary at the circularization radius
a$_{\rm c}$, and forms a circumbinary disk
\citep{2002apa..book.....F}.  Using conservation of angular momentum,
we equate the specific angular momentum of the accreted mass at the
inner Lagrangian point of the (outer) donor star to the final specific
angular momentum of the accretion stream at the circularization radius
about the inner binary, this results in
\begin{equation}
\label{eqn:specangmom1}
v_{\rm orb,3}(a_{\rm out} - R_{\rm L}) = v_{\rm orb,c}a_{\rm c},
\end{equation}
where R$_{\rm L}$ is the radius of the Roche lobe of the outer
tertiary companion, $a_{\rm c}$ is the semi-major axis of the orbit
about the inner binary corresponding to the circularization radius and
v$_{\rm orb,c}$ is the orbital velocity at $a_{\rm c}$.  The distance
from the centre of mass corresponding to the tertiary defined by the
Roche lobe is given by eq.\,2 in \cite{1983ApJ...268..368E}.
Combining eq.\,2 in \citet{1983ApJ...268..368E} (with mass ratio q
$=$ m$_{\rm 3}$/(m$_{\rm 1} +$m$_{\rm 2}$)) with
eq.~\ref{eqn:specangmom1}, we solve for the circularization
radius as a function of a$_{\rm out}$ and the assumed stellar masses:
\begin{equation}
\label{eqn:ac}
a_{\rm c} = a_{\rm out}(1 - R_{\rm L}).
\end{equation}
In order for a circumbinary disk to form around the inner binary, we
require that $a_{\rm in} < a_{\rm c}$.

Figure~\ref{fig:fig2} shows the parameter space in the P$_{\rm
  out}$-P$_{\rm in}$-plane for WOCS 7782.  Here we adopted, for
clarity, initial component masses of $m_{\rm 1} = 1.1$ M$_{\rm \odot}$
and $m_{\rm 2} = 0.9$ M$_{\rm \odot}$ for the inner binary components,
and $m_{\rm 3} = 1.4 $M$_{\rm \odot}$ for the outer tertiary.  The
scenario worked out is general, but we opt for these specific
parameters because they appear to naturally result in a system with
parameters similar to WOCS 7782.  We compare the circularization
radius to the semi-major axis of the inner binary, for which we
require $a_{\rm c} > a_{\rm in}$, after folding in all constraints
from the requirements for dynamical stability (listed in the caption
of figure~\ref{fig:fig2}), and the assumption of an outer tertiary
that is Roche lobe-filling (see \citet{2014MNRAS.438.1909D} for more
details).
%\simon{Why do you consider it important to mention that the orbit is hard, and for what environment did you consider this? - NL: Deleted.}  
Note that the range of plotted orbital periods P$_{\rm in}$
corresponding to a contact state for the inner binary lies outside the
range of plotted values for $P_{\rm in}$ (for components with radii of
1 R$_{\odot}$), since it does not contribute to constraining the outer
orbital properties.  The thick horizontal solid red line shows the
allowed range of outer semi-major axes, after folding in all of the
aforementioned criteria.  These constraints result in a rather narrow
range of initial conditions for the outer orbit, namely 2.2 $\times$
10$^{2}$ days $\le$ P$_{\rm out}$ $\le$ 1.1 $\times$ 10$^3$ days,
which also directly translates into constraints on the final
  outer tertiary orbit.
%, for
%which the tertiary star overfills its Roche lobe, the triple is (at
%least initially) dynamically stable and the accretion stream forms a
%gaseous ring around the inner binary.

Finally, we compute the timescales for our hypothesized triple to
undergo a direct interaction with another single or binary star.
Using the same assumptions for the host cluster properties of NGC 188
outlined in Section 3.2 of \citet{2011MNRAS.410.2370L} (right-hand
column), we find upon setting the single-triple (3+1) and
binary-triple (3+2) timescales equal to the expected duration of the
mass transfer phase (i.e., $\sim 1$\, Myr) critical outer orbital
periods for triples for $\gg 1$\,Myr.  These critical outer tertiary
orbital periods, which correspond to the times for a specific triple
to undergo an interaction, correspond to 3+1 and 3+2 interaction times
that are much longer than the maximum predicted outer orbital period
of the hypothesized white dwarf tertiary in our scenario.  We
therefore do not expect the mass transfer process to be interrupted by
a dynamical interaction in the cluster center.

\begin{figure}[ht!]
%\plotone{fig_parameterspace.eps}
\includegraphics[width=\columnwidth]{fig_parameterspace.pdf}
\caption{Parameter space in the P$_{\rm out}$-P$_{\rm in}$-plane
  allowed for the hypothetical outer tertiary orbit of WOCS 7782
  before Roche-lobe overflow.  The solid diagonal black line shows the
  period corresponding to the circularization radius a$_{\rm c}$ for
  the mass transfer stream coming from the outer star (i.e., at the
  onset of mass transfer).  
%, assuming a mass ratio of q $=$ 0.7.
% and
%  component masses of m$_{\rm 1} =$ 0.9 M$_{\odot}$, m$_{\rm 2} =$ 1.1
%  M$_{\odot}$ and m$_{\rm 3} =$ 1.4 M$_{\odot}$.  
We assume initial component masses of m$_{\rm 1}$ = 1.1 M$_{\rm
  \odot}$ and m$_{\rm 2}$ = 0.9 M$_{\rm \odot}$ for the inner binary
components, and m$_{\rm 3}$ is computed for the outer tertiary
according to our assumed mass ratio (with our fiducial case
corresponding to q $=$ 0.7).  We assume completely conservative mass
transfer for this exercise, and a final mass for the outer tertiary of
0.6 M$_{\rm \odot}$ once it has become a WD.  The dashed diagonal
black line shows a rough criterion for dynamical stability in the
triple, approximately following \citet{1999ASIC..522..385M} (i.e.,
a$_{\rm in} <$ 0.1a$_{\rm out}$ is required for long-term dynamical
stability in equal-mass co-planar triples).  The vertical solid red
lines show the outer orbital periods corresponding to the hard-soft
boundary assuming central velocity dispersions of $\sigma =$ 1, 5 and
10 km s$^{-1}$.  The vertical dashed black line show the maximum
outer orbital period P$_{\rm out}$ for which the outer tertiary
companion is Roche lobe-filling, assuming a stellar radius of R$_{\rm
  3} =$ 200 R$_{\odot}$ (which corresponds to the maximum stellar
radius reached on the AGB for the range of tertiary masses of interest
to us; see figure~\ref{fig:tertiarymass_vs_size}).  The horizontal
dashed red line shows the observed orbital period for WOCS 7782,
using its observed orbital period and our assumed final inner
companion masses (i.e., m$_{\rm 1} =$ m$_{\rm 2} =$ 1.4 M$_{\odot}$).
Finally, the thick solid horizontal red line shows the parameter space
for P$_{\rm out}$ allowed after considering all of the aforementioned
criteria.
\label{fig:fig2}}
\end{figure}

\begin{figure}[ht!]
  \includegraphics[width=\columnwidth]{fig_M14MSun_outerorbit.pdf}
  \caption{Giant radius as a function of the mass of the core of the
    Roche-lobe filling outer star (dark blue curve).  The first and
    last parts of this curve are orange to indicate that the star at
    these masses and radii is on the Hertzsprung-gap (to the left) or
    after core helium burning stage (to the right).  When the
    donor is on the giant branch (dark blue) Roche-lobe overflow will
    lead to a binary blue straggler.  Here we adopt a donor mass of
    1.4\,\MSun, but for an 1.2\,\MSun\, the donor the curve is quite
    similar (see dotted dark-blue curve).  The red squares in the
    curve show the parameters for which we performed more detailed
    gravitational-hydrodynamical simulations (see \S\,\ref{sims}).
    The horizontal dashed line shows the orbital separation of the
    observed twin BS 7782.  The initial triple in which it possibly
    formed must at least have been dynamically stable. The minimal
    orbital separation for the inner binary for which the triple is
    stable is given by the lower green coloured curve.  Donors which
    are smaller than about 100\,\RSun\, (light-green curve indicated
    with a$_{\rm stable}$) result in a dynamically unstable
    triple. The minimal core mass associated with a stable triple is
    then indicated by the left-most vertical dotted line.  The orbital
    separation at which the donor star overfills its Roche lobe is
    indicated with the light-blue curve. The top curve (brown) shows
    an estimate of the final orbital separation of the outer star, and
    therefore of the final orbit of the WD around the inner twin BSs.
    For core masses $\apgt 0.5$\,\MSun\, the final orbital separation,
    after mass transfer, is smaller than the initial orbit.  Here we
    adopted an initial inner binary mass of (1.0+0.9)\,\MSun\, and a
    final twin BS mass of (1.4+1.4)\,\MSun.
\label{fig:tertiarymass_vs_size}}
\end{figure}

Adopting a mass for the tertiary star of $m_3 = 1.4$\,\MSun, we can
constrain the initial parameters for the inner binary as well as the
orbit of the outer star after mass transfer. We first calculate the
stellar radius as a function of core mass. In
figure\,\ref{fig:tertiarymass_vs_size} we present this relation
calculated using the {\tt SeBa} stellar evolution code
\citep{1996A&A...309..179P} as the dark blue curve.  The interruption
in this curve, around a core mass of $m_{\rm core} \sim 0.5$\,\MSun\,
is a result of the evolution along the horizontal branch, where the
core of the star continues to grow but the radius actually shrinks.
Roche-lobe overflow in this phase is not expected to happen, because
it would already have happened in an earlier evolutionary state of the
donor star, when it was bigger.

Adopting masses for the inner binary $m_1=1.1$\,\MSun\, and
$m_2=0.9$\,\MSun\, we can calculate the outer orbital separation at
the onset of Roche-lobe overflow $a_{\rm out}$, and subsequently the
maximum orbital separation for the inner binary for which the orbit is
stable and a circumbinary disk can form. These two limits are
presented as the light blue and light green curves in
figure\,\ref{fig:tertiarymass_vs_size}.  The allotted region of
parameter space is then above the dashed horizontal line and to the
right of the vertical dotted line.

With the adopted parameters, we can also estimate the final orbital
period of the left-over core from the tertiary star after mass
transfer.  The change in orbital separation due to non-conservative
mass transfer can be expressed in terms of the mass of the outer star
before and after mass transfer, i.e. $m_3$ and $m'_3$ respectively,
the total mass in the inner binary before ($m_{\rm in}$) and after
accretion ($m'_{\rm in}$) and the amount of angular momentum lost per
unit mass $\eta \simeq 3$.  The value of $\eta = 3$ was derived
  in \cite{1991A&A...241..419P,1995A&A...296..691P} by matching the
  orbital evolution and birthrate of Be-type x-ray binaries that
  experience non-considervative mass transfer. This value of
  consistent with the analysis for non-conservative evolution of type
  B mass transfer in \cite{2011A&A...527A..84K} and further
  constrained in \cite{2007ASPC..367..387P} to understand the mass
  transfer in the 100-day orbital period binary V379 Cep.  Adopting
the relation between the orbital separation before mass transfer ($a$)
and after mass transfer ($a'$) from \cite{1995A&A...296..691P}
\begin{equation}
  {a' \over a} = \left( {m_3 m_{\rm in} \over m'_3 m'_{\rm in}} \right)^{-2}
  \left( {m_3 + m_{\rm in} \over m'_3 + m'_{\rm in}} \right)^{2\eta + 1},
\end{equation}
we arrive at the top red curve in
figure\,\ref{fig:tertiarymass_vs_size}. This curve provides a prediction for 
the current orbital separation of the WD around the twin
BS 7782.
%% Tidal effects during mass \textbf{transfer}
%% have probably circularized the orbit, although some slight eccentricity
%% due to turbulent motion in the outer layers of the donor star may have
%% induced a small $e \ll 0.1$ eccentricity.

Having limited parameter space for the formation of the twin BS 7782, we continue by performing a series of simulations to
investigate the accretion and changes to the inner orbits of triple
systems in this range of parameters.

\section{Numerical Simulations} \label{sims}

We perform simulations of a triple star system for which the outer
star overfills its Roche lobe while the inner binary remains
detached. The calculations start by evolving the three stars to the
same age, which is selected such that the outer-most star fills its
Roche lobe.  First order constraints for the initial conditions are
derived in the previous \S. In the following two sections we describe
how we set up these simulations and then discuss the results. The
calculations are performed using the Astrophysical Multipurpose
Software Environment using a combination of stellar evolution,
gravitational dynamics and hydrodynamics.

\subsection{Setting-up the simulations}

We adopt initial masses of $m_{\rm 1} = 1.1$ M$_{\rm \odot}$ and
$m_{\rm 2} = 0.7$ M$_{\rm \odot}$ or 0.9\,M$_{\rm \odot}$ for the
inner binary components, and between $m_{\rm 3} = 1.2$ and $m_{\rm 3}
= 1.4$ M$_{\rm \odot}$ for the tertiary star.  We evolve the tertiary
star using the MESA stellar-evolution code \cite{2011ApJS..192....3P}
to a radius of about 100\,\RSun\, and 150\,\RSun, at which point we
assume it to overfill it's Roche lobe (see red squares in
figure~\ref{fig:tertiarymass_vs_size}).  We perform calculations for
an inner orbital separation of $a_{\rm in} = 0.10$\,au, $a_{\rm in} =
0.15$\,au and $a_{\rm in} = 0.20$\,au.  In total we performed 12
calculations at a resolution of 40k SPH particles and 12 at 80k.

The stellar-evolution model, including the structure, temperature and
composition profiles are turned into a smoothed-particles
representation using the module {\tt StellarModelInSPH} in AMUSE (see
chapter 4 in \cite{AMUSE}).  We follow the same procedure as described
in \cite{2014MNRAS.438.1909D} for simulating the future of the triple
system $\chi$ Tau (HD 97131) in which the outer-most star overfills
its Roche lobe and transfers mass to an inner binary.  After
generating the hydrodynamical representation of the donor star we
replace the stellar core by a point mass to prevent the majority of
the resolution to be confined in the star's central regions.  In a
following step we relax the star using the hydrodynamics solver. This
relaxation process is realized in 100 steps during which we reduce the
velocity dispersion of individual SPH particles to a glasses structure 
%\citep[see, for example, \S\,3.3 on page 40 in][]{1995LesHouchesSWhite}. 
(see, for example, \S\,3.3 on page 40 in White (1995)).  During this procedure, the gaseous
envelope of the star tends to expand by about 20\%.  To determine the
radius of the evolving star we calculate Lagrangian radii and use the
distance to the stellar center which contains 90\% of its mass. From
this 90\% mass-radius relation we obtain the stellar radius and match
it with the Roche-lobe of the outer orbit.

With these parameters the orbital separation of the outer binary
becomes $\sim 250$ R$_{\rm \odot}$ for the 100 R$_{\rm \odot}$ donor star and about
430\,\RSun\, for the more evolved donor star.  We adopt the outer
orbit to be circular and in the plane of the inner binary.
%% We
%% performed several additional calculations for inclined orbits (choosing
%% a relative initial inclination $i = 9^\circ$, $i = 45^\circ$, $i =
%% 90^\circ$ and $i = 180^\circ$, but we wi
%% The low inclination is motivated by observed stellar triples composed
%% of low-mass stars to favour co-planar configurations
%% \citep{2018ApJ...854...44M}.
%% To test the effect of highly inclined
%% and retrogade orbits we include the $i = 90^\circ$ and $i = 180^\circ$
%% cases.
%%SPZ: I removed the notion of low-inclination, because we already said as such in the intro.
%% In figure~\ref{fig:topview_at_t0} we present a top view of the initial
%% conditions for one of these calculations.

%% \begin{figure}[ht!]
%%   \includegraphics[width=\columnwidth]{fig_BBSS_gas_M14Msun_A010au_t0002.png}
%% \caption{Top view of the simulated triple system in which the
%%   100\,\RSun\, outer star of 1.4\,\MSun\, over-fills its
%%   Roche-lobe.  The star is represented by 80000\,SPH particles and a
%%   core particle of $\sim 0.4$\,\MSun (black bullet). The two companion
%%   stars are represented as black bullets (to the right).  The inner
%%   binary is represented by the yellow and red bullets for, respectively, the
%%   1.1\,\MSun\, primary and 0.9\,\MSun\, secondary stars in a circular
%%   orbit of 0.1\,au. The 1.4\,\MSun\, giant star is presented to the
%%   right in a circular orbit with semi-major axis $\sim 250$\,\RSun\, in the plane of the
%%   inner binary.
%% \label{fig:topview_at_t0}}
%% \end{figure}

Roche-lobe overflow in triples is modelled using a coupled integrator
to follow the complex hydrodynamics of mass transfer from the
Roche-lobe filling outer star to the inner binary, while keeping track
of the gravitational dynamics of the stars.  The equations of motion
of the inner binary are solved using the symplectic direct N-body
integrator \texttt{Huayno} \citep{2012NewA...17..711P}. The
hydrodynamics are performed with the smoothed-particles hydrodynamics
code \texttt{Gadget2} \citep{2000ascl.soft03001S}, using an adiabatic
equation of state.  The two inner binary stars are treated as point
masses, but we allow them to accrete mass and angular momentum from
the gas liberated by the outer star.  This is realized using spherical
sink-particles that co-move with the mass points in the gravity
code. While the inner two stars accrete mass, they also accrete the
corresponding amount of angular momentum from the gas (see chapter 5
in \cite{AMUSE}).  The N-body integrator correctly accounts for this.
For the radius of the sink particles, we adopt $2 R_\odot$ for both
stars.

The N-body code, as well as the hydrodynamics solver, operate using
their own internal time-steps. The coupling between the two codes is
realized using the \texttt{Bridge} method in the AMUSE framework
\citep[see Sect.\.4.3.1 in][]{2013CoPhC.183..456P}.  This coupled
integrator is based on the splitting of the Hamiltonian, much in the
same way as is done with two different gravity solvers by
\cite{2007PASJ...59.1095F}. With the adopted scheme, the
hydrodynamical solver is affected by the gravitational potential of
its own particles, as well as the gravitational potential of the inner
binary. The hydrodynamics affects the orbits of the two inner stars
and the accretion onto the two stars affects the hydrodynamics. With
\texttt{Bridge} we realize a second order coupling between the
gravitational dynamics and the hydrodynamics.  The interval at which
the gravity and hydrodynamics interact via \texttt{Bridge} depends on
the parameters of the system we study, but typically we achieve
converged solutions when this time step is about 1/100 that of the
inner binary orbital period.

\section{Results of the hydrodynamical simulations} \label{results}

To test the hypothesis that the secondary in the inner binary
accretes more effectively than the primary star and to measure the
change to the inner orbit due to the Roche-lobe overflow of the outer
star, we perform a series of calculations in which we take the self
gravity and the hydrodynamical effects of the triple into account.
The results of one of these simulations (1091 days after
the onset of mass transfer) is presented in
figure~\ref{fig:mass_vs_semimajor_axis}.
%% figure~\ref{fig:topview_at_t1000day} and
%% The first figure
%% (figure~\ref{fig:topview_at_t1000day}) shows the top view of the same
%% initial realization for which we presented the initial conditions in
%% figure~\ref{fig:topview_at_t0} but now at an age of 1091 days after
%% the onset of mass transfer. We add, to the left panel,
%% the equipotential surfaces in the orbital plane.

It is apparent that the mass transfer in the adopted triples leads to
a rather untidy evolution, since much of the donor mass is lost
through the second Lagrangian point to the right side of the donor
star in figure~\ref{fig:topview_at_t1000day}. A considerable amount of
mass is also lost through the third Lagrangian point (to the left of
the inner binary), although it is hard to actually quantify the amount
of material lost, becuase an appreciable fraction may to rain back
onto the triple system.  One remaining question is how much mass is
eventually ejected altogether from the triple system and is therefore
not accreted to any of the two inner stars.  In our simulations the
accretion efficiency on the inner binary has to exceeds $\sim 80$\,\%
for the two blue straggers to reach masses comparable to those
observed in WOCS~ID~7782.  Over the relatively short time scale for
which we performed the calculations this efficiency is achieved, but
it is not clear how the system responds at later stages.

\begin{figure*}[ht!]
  \includegraphics[width=0.5\linewidth]{fig_BBSS_Lgas_M14Msun_A010au_t3360.png}
~  \includegraphics[width=0.5\linewidth]{fig_BBSS_gas_M14Msun_A010au_t3360.png}
  \caption{ Top view of one of the simulated triple systems at an age
    of $t \simeq 1091\,days$ after the start of the simulation when
    the 100\,\RSun\, outer star of 1.4\,\MSun\, over-fills its
    Roche-lobe.  The star is represented by 80000\,SPH particles and a
    core particle of $\sim 0.4$\,\MSun (black bullet in the middle of
    the right-most yellow blob). The inner binary (to the
      left) is represented by the yellow and red bullets for,
    respectively, the 1.1\,\MSun\, primary and 0.9\,\MSun\, secondary
    stars in a circular orbit of 0.1\,au. The 1.4\,\MSun\, giant star
    is presented to the right in a circular orbit with semi-major axis
    $\sim 250$\,\RSun\, in the plane of the inner binary.  Left panel
    shows the equipotential surfaces of the triple overplotted with
    the gas distribution, the right panel shows just the gas and the
    stars as bullets.
\label{fig:topview_at_t1000day}}
\end{figure*}

The evolution of the inner orbit presented for several simulations in
figure~\ref{fig:mass_vs_semimajor_axis} is complicated.  This is
caused by the complex transport of mass, energy and angular momentum
through the accretion stream and throughout the system.  It is
therefore hard to quantify distinct trends in the evolution of the
triple system. In simulations of the response of an inner binary on
accretion from a circumbinary disk, \cite{2018arXiv181208175M}
conclude that the complexity of angular momentum transport between the
outer star and the accretion stream onto the individual inner stars,
is complicated and without clear trends. For most of our calculations
we agree with this statement, but in
figure~\,\ref{fig:mass_vs_semimajor_axis} we nevertheless present the
results of 6 of our calculations, three for a 1.2\,\MSun\, donor star
and three for a 1.4\,\MSun\, donor. The various coloured curves give
the resulting evolution of the inner orbit as a function of the total
mass in the inner binary. As the inner two stars accrete, the orbit
shrinks for a 1.2\,\MSun\, donor. These systems are expected to result
in a contact binary, that eventually may merge to form a single BS
with a mass more than twice the turn off in orbit around a low-mass
white dwarf. The required evolution in order to explain the observed
twin BS 7782 is indicated by the three black curves; the simulated
path clearly deviates from these. We, therefore, argue that a
1.2\,\MSun\, donor has difficulty explaining the observed orbital
separation of $\sim 0.13$\,au in BSS 7782.

\begin{figure*}[ht!]
  \includegraphics[width=0.51\linewidth]{fig_orbital_evolution_for_12MSun.pdf}
~  \includegraphics[width=0.49\linewidth]{fig_orbital_evolution_for_14MSun.pdf}
  \caption{Evolution of the orbital separation as a function of the
    total mass of the inner binary for six calculations with somewhat
    different initial conditions (see the legend). The left panel
    shows the result for a 1.2\,\MSun\, donor star and the right panel
    for a 1.4\,\MSun\, donor. The initial binary shown by the blue
    curve of the right-hand panel is presented in
    fig.\,\ref{fig:topview_at_t1000day} we present the final
    conditions of this system.  The black curves give the expected
    evolution of the orbital separation of the inner binary assuming
    that the binary evolved towards the observed orbital separation of
    0.13\,au at a total binary mass of 2.8\,\MSun.
\label{fig:mass_vs_semimajor_axis}}
\end{figure*}

In the right-hand panel in figure~\ref{fig:mass_vs_semimajor_axis} we
present the evolution of the orbit for the 1.4\,\MSun\, donor for
several initial orbits of the inner binary. A more massive donor
appears to be more effective in producing a twin BS with parameters
consistent with the observed system 7782. There is more mass available
in the envelope of the donor star, and the orbital evolution of the
inner binary matches better with the anticipated evolution
needed to reproduce the observed parameters of WOCS 7782.  A
more massive donor may therefore have a lower accretion efficiency
while still accomodating the observed constraints.  The longer thermal
time scale of the stellar envelope of the higher-mass donor at the
same stellar radius eventually leads to a higher mass-transfer rate,
and therefore to a lower accretion efficiency. However, the larger
mass budget in the envelope appears to compensate.

The orbit of the inner binary expands in these cases as a result of
accretion onto the inner two stars. In all three cases for the
1.4\,\MSun\, donor presented in
figure~\ref{fig:mass_vs_semimajor_axis} the inner orbit expands at
about the same rate. Consequently, the inner binaries that start with
$a = 0.15$\,au and $a=0.20$\,au eventually become dynamically
unstable.  The binary with an initial separation of $0.10$\,au expands
to reach a separation of about 0.126--0.145\,au for final masses for
the inner two stars of 1.4\,\MSun, which is consistent with the
observed twin BS 7782. In our simulations the eccentricity of the
inner binary grows to about $e \simeq 0.0028$.

With the accretion of mass, both stars in the inner binary also
accrete angular momentum.  By the end of the simulation the spins of
the two BSs are aligned along the orbital angular momentum axis with
an angle of $90.0^\circ$ for the primary star and $93.4^\circ$ for the
secondary star with respect to the argument of pericenter of the inner
orbit.  This supports our naive prediction that the spin angular
momenta of the two stars in the inner binary should be more or less
aligned post-mass transfer, due to the non-negligible amount of mass
accreted. By the end of the simulations the spin of the primary is
about 50.5 rotations per day, and 41.5 rotations per day for the
secondary star.  Such high spin rates immediately post-mass transfer
are supported by other work (see, for example,
\citet{2014MNRAS.438.1909D}), but the twin blue stragglers in
WOCS~ID~7782 are not observed to be spinning that fast
\cite{2018ApJ...869L..29L}.  Rapdily spinning stars may slow due to,
for example, magnetic braking, bringing them closer to the actually
observed spin rates \cite{2018arXiv181202181L}.
  
%% But by this time both stars have accreted
%% only a small portion of mass and we expect that the final spins of
%% both stars will be close to break-up (ignoring any spin-down from
%% magnetic braking effects).

\section{Discussion} \label{sect:discussion}

In this paper, we propose a formation scenario for twin equal-mass
blue stragglers in tight binaries, as observed for WOCS 7782 in the
old OC NGC 188.  The proposed scenario involves mass transfer from an
evolved outer tertiary companion. Part of this mass is accreted by the
inner binary via a circumbinary disk, while the rest escapes through the
second and third Lagrangian points in the potential of the triple
system.  Our scenario makes several predictions for the observed
properties of a hypothetical outer triple companion, now a WD.  These
are:

\begin{enumerate}

\item For the predicted outer tertiary orbit, the initial orbital
  period should lie between 220 days $\lesssim$ P$_{\rm out}$
  $\lesssim$ 1100 days, assuming initial masses for the inner binary
  components of $m_{\rm 1} = 1.1$ M$_{\odot}$ and $m_{\rm 2} = 0.9$
  M$_{\odot}$ and an initial outer tertiary mass of $m_{\rm 3} = 1.4
  $M$_{\odot}$. The final orbital period of the white dwarf
    around the binary blue straggler should exceeds the initial
    orbit, but be smaller than $\sim 4100$\,days.

\item Larger final WD masses, and hence larger core masses for the
  donor at the time of mass transfer should correspond to larger final
  outer orbital periods for the tertiary.  This is because the Roche
  radius is larger for larger outer orbital periods, such that the
  donor must evolve to larger radii, and hence core masses, before the
  onset of mass transfer. We expect the orbital separation to range
  from $\apgt 6.4$\,yr for a $\sim 0.42$\,\MSun\, white dwarf to
  $\apgt 11.2$\,yr for a $\sim 0.48$\,\MSun\, white dwarf.

\item For the inner binary, the rotational axes of both the BSs should
  be aligned with each other and the orbital plane of the outer
  tertiary WD.  This is because accretion onto the BS progenitors
  proceeds via an accretion disk, that forms at the circularization
  radius and that has an orbital plane aligned with that of the outer
  tertiary.

\item The BSs in the inner binary should have roughly equal masses,
  independent of their initial masses.  This is because it is the
  lowest mass object that typically accretes the fastest, since its
  orbital velocity and distance relative to the circumbinary disk is
  typically the lowest
%% \citep[e.g.][]{1995MNRAS.277.1491K,1997MNRAS.285...33B,2000MNRAS.314...33B\
%%}.                                                                            
  \citep[e.g.][]{2000MNRAS.314...33B,2012ApJ...749..118S,2017MNRAS.466.1170M}.
  The mass ratio of the inner binary, therefore grows to unity.  

  We further validated this statement by performing an
    additional series of calculations in which we vary the mass of the
    tertiary star in the initial triple from 0.5\,\MSun\, to
    0.7\,\MSun\, and 0.9\,\MSun. In
    fig.\,\ref{fig:mass_ratio_evolution_for_14MSun} we present the
    evolution of the normalized mass ratio in these binaries. with
    these calculations we demonstrate that a low-mass ratio initially
    tends to evolve towards an equal mass ratio. 
  
  \begin{figure}[ht!]
  \includegraphics[width=\columnwidth]{fig_mass_ratio_evolution_for_14MSun.pdf}
  \caption{Evolution of the mass ratio for initial triples with an
    inner orbital separation of 0.1\,au (orange dotted curve) and
    0.2\,au (all other curves). The initial primary mass was
    1.4\,\MSun\, overfilling it's Roche lobe at a radius of
    100\,\RSun. the companion masses are $m_2= 0.9$\,\MSun\, and the
    mass of the tertiary is indicated in the legend. The dotted black
    line indicates the required mass-ratio evolution in order to
    eventually reach an equal mass-ratio blue-straggler binary.
\label{fig:mass_ratio_evolution_for_14MSun}}
\end{figure}

\end{enumerate}

Finally, we emphasize that the choice for the initial mass of the
outer tertiary may be rather critical.  Mass transfer in our proposed
scenario proceeds from the most massive tertiary to a binary of lower
total mass. This may result in an unstable phase of mass transfer, in
particular if the donor has a convective envelope
\citep[e.g.][]{2009pfer.book.....M}. A radiative envelope of the donor
ensures that the mass transfer will be maximally conservative, such
that the accretion stream will be maximally stable, accreting at a
stable and roughly constant rate \citep[e.g.][]{1991ApJS...76...55I}.
This stability regime may also be of interest for explaining very
massive twins, of $\apgt 20$\,\MSun\, which could be promising sources
for gravitational wave detectors once both twins evolve to a binary
black hole \citep{2016MNRAS.460.3545D}.

\section{Summary} \label{sect:conclusions}

In this paper, we consider the formation of twin BSs in tight binaries.
These systems may form through mass transfer from an outer Roche-lobe
filling tertiary star. Once this star ascends the giant branch, part
of its envelope is transferred to the inner binary, and accreted by
the two inner stars which are still on the MS.

As illustrated via SPH simulations, the mass transfer stream forms a
circumbinary disk, from which the inner binary stars accrete, driving
the inner binary toward a mass ratio close to unity.  Our simulations
indicate that the inner binary orbital separation can decrease or
expand depending on the details of the transfer of mass and angular
momentum.  More work is certainly needed in order to fully understand
mass transfer in triples.

We summarize the results of these simulations as follows: for a
1.2\,\MSun\, tertiary donor mass, we expect the inner two stars to
eventually merge and form a single BS. This reduces the system to a
binary with a primary BS and an outer WD in a relatively wide
orbit. Such a BS will distinguish itself from other BSs by potentially
being more than twice the turn-off mass in a star cluster.  An example
could be the $2.9\pm0.2$\,\MSun\, BS S1237 in the Galactic cluster M67
\citep{2016ApJ...832L..13L}. It is the primary of a $\sim 698$\,day
binary with an eccentric orbit of $\sim 0.10$.

With an original outer star of mass $\sim 1.4$\,\MSun, the inner orbit
tends to expand. This eventually leads to a dynamically unstable
system resulting either in a collision or in the ejection of
(probably) the lowest mass star. This evolution could result in a
single ejected BS, with the other BS left in a relatively close and
eccentric orbit with a WD (the left-over core of the tertiary star).
Such a close BS-WD binary would be hard to explain in another way.
%% These ``imposter'' BS-WD binaries would in principle mimic what is
%% expected theoretically for BSs formed from mass transfer in binary
%% stars.
If such a dynamical instability engages relatively late in the
mass-tranfer phase, the white dwarf (maybe with a little left-over
envelope) is expected to be ejected. This would lead to a relatively
wide twin blue-straggler binary and a single low-mass white dwarf.

When we adopt an inner orbit of $0.10$\,au\, the expansion eventually
matches the observed orbital separation (i.e., $0.13$\,au) of the
observed twin BS 7782 and the observed masses of the two stars of
about 1.4\,\MSun.

In order to study the T-tauri binaries V4046 Sgr and DQ Tau,
\cite{2011MNRAS.413.2679D} perform a series of 2D hydrodynamical
simulations of circumbinary disks.  These authors studied the two
observed T-tauri systems V4046 Sgr and DQ~Tau, to which we compare our
results here.  For V4046 Sgr, for which the two stars have comparable
masses as in our calculation for a circular orbit with a period of
only 2.4 days, they find that the inner binary accretes at a rate of
$\sim 0.028$\,\MSun/Myr.  For DQ~Tau, which is composed of lower-mass
stars ($m_1 = m_2 \simeq 0.55$\,\MSun) in an eccentric ($e\simeq              
0.556$) orbit of $\sim 15.8$\,days, they find an accretion rate onto
the inner binary of $\sim 0.027$\,\MSun/Myr.  These values are in the
same range as in our calculations, which results in an accretion rate
for the inner binary of 0.027--0.058\,\MSun/Myr (i.e., the average
measured over a period of about 3000 days in our simulations).
Interestingly, however, \cite{2011MNRAS.413.2679D} find that the
primary star in V4046 Sgr accretes at an 8\% higher rate than the
secondary star, whereas in our case the secondary star accretes at a
higher rate than the primary star by about 1\% to 12\%.  Higher
accretion rates in the secondary star are realized for eccentric and
retrogade inner orbits. We performed an extra series of calculations
to further study this, but they all lead to the merger of the inner
binary.




\acknowledgments

The authors kindly thank two anonymous reviewers for their considerable insight and 
suggestions for improvement.  N.W.C.L. acknowledges support from a Kalbfleisch Fellowship at the
American Museum of Natural History.  SPZ would like to thank Norm
Murray and CITA for their hospitality during my long-term visit.  This
work was supported by the Netherlands Research School for Astronomy
(NOVA). 
%%
In this work we use the matplotlib
\citep{2007CSE.....9...90H}, numpy
\citep{Oliphant2006ANumPy}, AMUSE
\citep{portegies_zwart_simon_2018_1443252}, SeBa
\citep{2012ascl.soft01003P}, Huayno \citep{2012NewA...17..711P}, MESA
\citep{2010ascl.soft10083P}, and GadGet2 \citep{2000ascl.soft03001S}
packages. The calculations ware performed using the LGM-II (NWO grant
\# 621.016.701) and the Dutch National Supercomputer at SURFSara
(grant \# 15520).

\bibliographystyle{apj}
\bibliography{BSS7782}


%\input /home/spz/Latex/lib/bib/references
%\bibliographystyle{/home/spz/Latex/lib/styles/elsevier/elsarticle-num} 
%\bibliography{references}      


\end{document}
