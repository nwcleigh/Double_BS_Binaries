%%
%% Beginning of file 'sample62.tex'
%%
%% Modified 2018 January
%%
%% This is a sample manuscript marked up using the
%% AASTeX v6.2 LaTeX 2e macros.
%%
%% AASTeX is now based on Alexey Vikhlinin's emulateapj.cls 
%% (Copyright 2000-2015).  See the classfile for details.

%% AASTeX requires revtex4-1.cls (http://publish.aps.org/revtex4/) and
%% other external packages (latexsym, graphicx, amssymb, longtable, and epsf).
%% All of these external packages should already be present in the modern TeX 
%% distributions.  If not they can also be obtained at www.ctan.org.

%% The first piece of markup in an AASTeX v6.x document is the \documentclass
%% command. LaTeX will ignore any data that comes before this command. The 
%% documentclass can take an optional argument to modify the output style.
%% The command below calls the preprint style  which will produce a tightly 
%% typeset, one-column, single-spaced document.  It is the default and thus
%% does not need to be explicitly stated.
%%
%%
%% using aastex version 6.2
\documentclass{aastex62}

%% The default is a single spaced, 10 point font, single spaced article.
%% There are 5 other style options available via an optional argument. They
%% can be envoked like this:
%%
%% \documentclass[argument]{aastex62}
%% 
%% where the layout options are:
%%
%%  twocolumn   : two text columns, 10 point font, single spaced article.
%%                This is the most compact and represent the final published
%%                derived PDF copy of the accepted manuscript from the publisher
%%  manuscript  : one text column, 12 point font, double spaced article.
%%  preprint    : one text column, 12 point font, single spaced article.  
%%  preprint2   : two text columns, 12 point font, single spaced article.
%%  modern      : a stylish, single text column, 12 point font, article with
%% 		  wider left and right margins. This uses the Daniel
%% 		  Foreman-Mackey and David Hogg design.
%%  RNAAS       : Preferred style for Research Notes which are by design 
%%                lacking an abstract and brief. DO NOT use \begin{abstract}
%%                and \end{abstract} with this style.
%%
%% Note that you can submit to the AAS Journals in any of these 6 styles.
%%
%% There are other optional arguments one can envoke to allow other stylistic
%% actions. The available options are:
%%
%%  astrosymb    : Loads Astrosymb font and define \astrocommands. 
%%  tighten      : Makes baselineskip slightly smaller, only works with 
%%                 the twocolumn substyle.
%%  times        : uses times font instead of the default
%%  linenumbers  : turn on lineno package.
%%  trackchanges : required to see the revision mark up and print its output
%%  longauthor   : Do not use the more compressed footnote style (default) for 
%%                 the author/collaboration/affiliations. Instead print all
%%                 affiliation information after each name. Creates a much
%%                 long author list but may be desirable for short author papers
%%
%% these can be used in any combination, e.g.
%%
%% \documentclass[twocolumn,linenumbers,trackchanges]{aastex62}
%%
%% AASTeX v6.* now includes \hyperref support. While we have built in specific
%% defaults into the classfile you can manually override them with the
%% \hypersetup command. For example,
%%
%%\hypersetup{linkcolor=red,citecolor=green,filecolor=cyan,urlcolor=magenta}
%%
%% will change the color of the internal links to red, the links to the
%% bibliography to green, the file links to cyan, and the external links to
%% magenta. Additional information on \hyperref options can be found here:
%% https://www.tug.org/applications/hyperref/manual.html#x1-40003
%%
%% If you want to create your own macros, you can do so
%% using \newcommand. Your macros should appear before
%% the \begin{document} command.
%%
\newcommand{\vdag}{(v)^\dagger}
\newcommand\aastex{AAS\TeX}
\newcommand\latex{La\TeX}

\newcommand{\MSun}{\mbox{M$_\odot$}}
\newcommand{\RSun}{\mbox{R$_\odot$}}
\newcommand{\LSun}{\mbox{L$_\odot$}}

\def\apgt{\ {\raise-.5ex\hbox{$\buildrel>\over\sim$}}\ }
\def\aplt{\ {\raise-.5ex\hbox{$\buildrel<\over\sim$}}\ }

\def\Simon#1{{\bf {\color{red}[#1 -- Simon]}}}
\def\simon#1{{\bf {\color{red}[#1 -- Simon]}}}
\def\del#1{{\bf {\sout{#1}}}}
\def\replace#1#2{{\bf {\sout{#1} $\rightarrow$ {\bf #2}}}}


%% Reintroduced the \received and \accepted commands from AASTeX v5.2
\received{January 1, 2018}
\revised{January 7, 2018}
\accepted{\today}
%% Command to document which AAS Journal the manuscript was submitted to.
%% Adds "Submitted to " the arguement.
\submitjournal{ApJ}



%% Mark up commands to limit the number of authors on the front page.
%% Note that in AASTeX v6.2 a \collaboration call (see below) counts as
%% an author in this case.
%
%\AuthorCollaborationLimit=3
%
%% Will only show Schwarz, Muench and "the AAS Journals Data Scientist 
%% collaboration" on the front page of this example manuscript.
%%
%% Note that all of the author will be shown in the published article.
%% This feature is meant to be used prior to acceptance to make the
%% front end of a long author article more manageable. Please do not use
%% this functionality for manuscripts with less than 20 authors. Conversely,
%% please do use this when the number of authors exceeds 40.
%%
%% Use \allauthors at the manuscript end to show the full author list.
%% This command should only be used with \AuthorCollaborationLimit is used.

%% The following command can be used to set the latex table counters.  It
%% is needed in this document because it uses a mix of latex tabular and
%% AASTeX deluxetables.  In general it should not be needed.
%\setcounter{table}{1}

%%%%%%%%%%%%%%%%%%%%%%%%%%%%%%%%%%%%%%%%%%%%%%%%%%%%%%%%%%%%%%%%%%%%%%%%%%%%%%%%
%%
%% The following section outlines numerous optional output that
%% can be displayed in the front matter or as running meta-data.
%%
%% If you wish, you may supply running head information, although
%% this information may be modified by the editorial offices.
\shorttitle{Compact Binaries of BS Twins from Stellar Triples}
\shortauthors{Leigh \& Portegies Zwart}
%%
%% You can add a light gray and diagonal water-mark to the first page 
%% with this command:
% \watermark{text}
%% where "text", e.g. DRAFT, is the text to appear.  If the text is 
%% long you can control the water-mark size with:
%  \setwatermarkfontsize{dimension}
%% where dimension is any recognized LaTeX dimension, e.g. pt, in, etc.
%%
%%%%%%%%%%%%%%%%%%%%%%%%%%%%%%%%%%%%%%%%%%%%%%%%%%%%%%%%%%%%%%%%%%%%%%%%%%%%%%%%

%% This is the end of the preamble.  Indicate the beginning of the
%% manuscript itself with \begin{document}.

\begin{document}

\title{A Triple Origin for Twin Blue Stragglers in Close Binaries}

%% LaTeX will automatically break titles if they run longer than
%% one line. However, you may use \\ to force a line break if
%% you desire. In v6.2 you can include a footnote in the title.

%% A significant change from earlier AASTEX versions is in the structure for 
%% calling author and affilations. The change was necessary to implement 
%% autoindexing of affilations which prior was a manual process that could 
%% easily be tedious in large author manuscripts.
%%
%% The \author command is the same as before except it now takes an optional
%% arguement which is the 16 digit ORCID. The syntax is:
%% \author[xxxx-xxxx-xxxx-xxxx]{Author Name}
%%
%% This will hyperlink the author name to the author's ORCID page. Note that
%% during compilation, LaTeX will do some limited checking of the format of
%% the ID to make sure it is valid.
%%
%% Use \affiliation for affiliation information. The old \affil is now aliased
%% to \affiliation. AASTeX v6.2 will automatically index these in the header.
%% When a duplicate is found its index will be the same as its previous entry.
%%
%% Note that \altaffilmark and \altaffiltext have been removed and thus 
%% can not be used to document secondary affiliations. If they are used latex
%% will issue a specific error message and quit. Please use multiple 
%% \affiliation calls for to document more than one affiliation.
%%
%% The new \altaffiliation can be used to indicate some secondary information
%% such as fellowships. This command produces a non-numeric footnote that is
%% set away from the numeric \affiliation footnotes.  NOTE that if an
%% \altaffiliation command is used it must come BEFORE the \affiliation call,
%% right after the \author command, in order to place the footnotes in
%% the proper location.
%%
%% Use \email to set provide email addresses. Each \email will appear on its
%% own line so you can put multiple email address in one \email call. A new
%% \correspondingauthor command is available in V6.2 to identify the
%% corresponding author of the manuscript. It is the author's responsibility
%% to make sure this name is also in the author list.
%%
%% While authors can be grouped inside the same \author and \affiliation
%% commands it is better to have a single author for each. This allows for
%% one to exploit all the new benefits and should make book-keeping easier.
%%
%% If done correctly the peer review system will be able to
%% automatically put the author and affiliation information from the manuscript
%% and save the corresponding author the trouble of entering it by hand.

\correspondingauthor{Nathan W. C. Leigh}
\email{nleigh@amnh.org}

\author{Nathan W. C. Leigh}
\affil{American Museum of Natural History \\
Department of Astrophysics \\
79th Street at Central Park West \\
New York, NY 10024-5192, USA}
\affil{Stony Brook University \\
Department of Physics and Astronomy\\
Stony Brook, NY 11794-3800, USA}
\affil{Departamento de Astronom\'ia \\ 
Facultad de Ciencias F\'isicas y Matem\'aticas \\ 
Universidad de Concepci\'on \\ 
Concepci\'on, Chile}


\author{Simon Portegies Zwart}
\affiliation{Leiden Observatory \\
Leiden University \\
PO Box 9513, 2300 RA \\
Leiden, the Netherlands}


%% Note that the \and command from previous versions of AASTeX is now
%% depreciated in this version as it is no longer necessary. AASTeX 
%% automatically takes care of all commas and "and"s between authors names.

%% AASTeX 6.2 has the new \collaboration and \nocollaboration commands to
%% provide the collaboration status of a group of authors. These commands 
%% can be used either before or after the list of corresponding authors. The
%% argument for \collaboration is the collaboration identifier. Authors are
%% encouraged to surround collaboration identifiers with ()s. The 
%% \nocollaboration command takes no argument and exists to indicate that
%% the nearby authors are not part of surrounding collaborations.

%\newcommand{\mnras}{\textrm{MNRAS}}
%\newcommand{\apj}{\textrm{ApJ}}
%\newcommand{\aap}{\textrm{A\&A}}
%\newcommand{\apjl}{\textrm{ApJ}}

\begin{abstract}

We propose a formation mechanism for twin blue stragglers in compact
binaries that involves mass transfer from an evolved outer tertiary
companion on to the inner binary via a circumbinary disk.  We apply
this scenario to the observed double blue straggler system Binary 7782
in the old open cluster NGC 188, and show that its observed properties
are naturally reproduced within the context of the proposed model.
Based on this model, we predict the following properties for twin blue
stragglers: (1) For the outer tertiary orbit, the initial orbital
period should lie between 220 days $\lesssim$ P$_{\rm out}$ $\lesssim$
1100 days, assuming initial masses for the inner binary components of
$m_{\rm 1} = 1.1$ M$_{\odot}$ and $m_{\rm 2} =$ 0.9 M$_{\odot}$ and an
outer tertiary mass of $m_{\rm 3} = 1.4$ M$_{\odot}$.  After
Roche-lobe overflow, the outer star turns into a white dwarf (WD) of
0.43 to 0.54\,\MSun. There is a correlation between the mass of this
WD and the outer orbital period in the sense that more massive white
dwarfs will be on wider orbits.  (3) The rotational axes of both the
BSs in the inner binary will be aligned with each other and the
orbital plane of the outer tertiary WD. (4) The BSs in the inner
binary will have roughly the same masses, independent of their initial
masses.  This follows from the more copious accretion of mass onto the
initially lower mass star.  This star should therefore be enriched
more effectively by the accreted material.  As a result, one of the
blue stragglers will appear to be more enriched by He, C and O
if the donor started to overflow its Roche lobe at the red
giant branch, or with s-process elements if the donor was on the
asymptotic giant branch.  (5) Relative to old dense clusters with high
velocity dispersions, twin BSs in close binaries formed from the
proposed mechanism should be more frequent in the Galactic field and
younger open clusters with ages $\lesssim$ 4-6 Gyr, since then the
donor will have a radiative envelope. (6) the orbit of the binary BS
will be close (typically $\aplt 0.3$\,au) and close to circular ($e
\aplt 0.2$).

\end{abstract}

\keywords{stars: blue stragglers -- binaries: general -- globular clusters: general -- scattering}


%%%%%%%%%%%%%%%%%%%%%%%%%%%%%%%
\section{Introduction} \label{intro}

Blue straggler (BS) stars are brighter and bluer than the
main-sequence turn-off in a cluster colour-magnitude diagram
\citep[e.g.][]{1953AJ.....58...61S,2014ApJ...782...49S}.  Two
primary channels for BS formation have been proposed: mass transfer
from an evolved donor on to a main-sequence star in a binary star
system
\citep[e.g.][]{1964MNRAS.128..147M,1997A&A...328..143P,2009Natur.457..288K,2011MNRAS.410.2370L,2011Natur.478..356G},
and direct stellar collisions involving main-sequence stars possibly
mediated via binaries
\citep[e.g.][]{hills75,1997A&A...328..130P,2007ApJ...661..210L,2013MNRAS.428..897L,2013MNRAS.429.1221H,2018arXiv181100058P}.
Other possible, albeit related, formation mechanism includes mergers
of close main-sequence binaries (MS-MS) \cite{2018arXiv181100058P},
and mergers of the inner binaries of hierarchical triple star systems
induced by Lidov-Kozai oscillations coupled with tidal damping
\citep[e.g.][]{2009ApJ...697.1048P}.

In spite of these specific predictions for the expected properties of
BSs formed from each of the above production mechanisms, many BSs
exist with observed properties that defy these simple scenarios.  For
example, in the old open cluster M67, there lurks a candidate triple
system that is posited to host two BSs
\citep{2001A&A...375..375V,2003AJ....125..810S}.  The observations
suggest that the outer tertiary is itself a BS, with a mass $\sim$ 1.7
M$_{\odot}$ and orbiting the inner binary with a period of $\sim
1188.5$ days \citep{2003AJ....125..810S}.  The inner binary has a
period of only $\sim 1.068$ days \citep{2001A&A...375..375V}, and
hosts a BS of mass $\sim 2.52$ M$_{\odot}$.  In order to form this
system we require at least five stars in order to conserve the total
mass in the system \citep{2011MNRAS.410.2370L}.  As explained in
\citet{2011MNRAS.410.2370L}, this is strongly indicative of a
dynamical origin for the system, and a single direct interaction
involving a binary and a triple that resulted in two separate
collision events is the most probable explanation for its origin
(i.e., a single interaction involving two multiples with two or more
stars is a more likely scenario to produce this system than two
back-to-back direct binary-binary interactions)
\citep{2004MNRAS.350..615G}.  \simon{Maybe add something about the
  curious binary AE Aurigae? - NL2: For sure.  I still don't know
  anything about it... But if I get time over the next few days I will
  do my research and write something.}

Even more curious, there exists in the old open cluster NGC 188 a
double BS binary, called Binary 7782.  More generally, the BS
population in NGC 188 has a bi-modal period-eccentricity distribution.
As discussed in \citet{2011MNRAS.410.2370L}, this could be hinting at a triple
origin for at least some subset of the total BS population.  As for
Binary 7782, \citet{2009Natur.462.1032M} observed a compact and mildly eccentric
(i.e., $e \sim 0.1$) binary star system with an orbital period of
$\sim$ 10 days hosting two roughly equal-mass blue stragglers.  During
a given binary-binary interaction, the probability that not one but
two direct (MS-MS) collisions will occur is less than $10^{-2}$
\citep{1989AJ.....98..217L,2011MNRAS.410.2370L,2012MNRAS.425.2369L}.  Plus, binaries with collision products
typically have relatively long orbital periods
\cite{2011Sci...334.1380F}. Dynamically, it is difficult to form a
short-period binary composed of two collision products during a
collisional interaction in a star cluster \citep{2011Sci...334.1380F}.
So, how did Binary 7782 form?

We propose a formation channel for Binary 7782, and compact double BS
binaries in general, which involves mass transfer from an outer
tertiary companion on to an inner (MS-MS) binary.  In
section~\ref{sect:dyn}, we constrain the range of initial (i.e., pre-mass
transfer) orbital parameters for a hypothetical outer tertiary
companion, using a combination of dynamical and stellar
evolution-based constraints.  In Section~\ref{sims} we present the
numerical simulations used to study the mass transfer process in our
hypothetical triple system, computed using the
\texttt{AMUSE}\cite{AMUSE} software package, in order to study the
evolution of the inner and outer orbital parameters during mass
transfer.  We summarize and discuss the implications of our results
for compact double BS binaries and, more generally, mass transfer in
stellar triples in Section~\ref{sect:discussion}.

\section{Constraints on the present-day orbital parameters for a hypothesized
         tertiary companion in the compact BS Binary 7782} \label{sect:dyn}

Consider a hierarchical triple system with component masses $m_{\rm
  1}$ and $m_{\rm 2}$ for the inner binary, and mass $m_{\rm 3}$ for
the outer tertiary companion.  The inner and outer binary orbital
semi-major axes are denoted a$_{\rm in}$ and a$_{\rm out}$,
respectively.  For clarity we now assume both orbits, the inner as
well as the outer, to have a small eccentricity and low inclination.
These assumptions are also supported by the population of observed
low-mass triples \cite{2010yCat..73890925T,moe18}.  This initial
configuration for our assumed formation scenario for Binary 7782,
described below, is depicted in Figure~\ref{fig:fig1}.

\begin{figure}[ht!]
\plotone{fig1.eps}
\caption{Cartoon depiction of our proposed scenario for the formation
  of Binary 7782, specifically mass transfer from an evolved outer
  tertiary companion on to a compact inner binary via a circumbinary
  disk.  The outer tertiary component has mass m$_{\rm 3}$, whereas
  the inner binary components have masses m$_{\rm 1}$ and m$_{\rm 2}$.
  The inner and outer orbital separations are denoted by,
  respectively, a$_{\rm in}$ and a$_{\rm out}$.  The circularization
  radius of the accretion stream is denoted a$_{\rm c}$, as calculated
  via Equation~\ref{eqn:ac}, and marks the mean separation of the
  circumbinary disk.\
  \simon{Should we also indicate the accretion stream onto the two stars? - NL2:  I have now updated this plot to show two circumstellar disks, and the flow on to thiose.  I can edit further, if you recommend it, but it's been tough with the down-sizing.  Maybe if I make things bigger still...  Anyways, happy to try more.}
\label{fig:fig1}}
\end{figure}

We consider a scenario where the outer star is filling its Roche lobe
and transfers mass to the inner binary.  The mass transfer stream
gathers at the circularization radius a$_{\rm c}$, and forms a circumbinary disk
around the inner binary.  Using conservation of angular momentum, we
equate the specific angular momentum of the accreted mass at the inner
Lagrangian point of the (outer) donor star to the final specific
angular momentum of the accretion stream at the circularization radius
about the inner binary, this results in
\begin{equation}
\label{eqn:specangmom1}
v_{\rm orb,3}(a_{\rm out} - R_{\rm L}) = v_{\rm orb,c}a_{\rm c},
\end{equation}
where R$_{\rm L}$ is the radius of the Roche lobe of the outer
tertiary companion, $a_{\rm c}$ is the semi-major axis of the orbit
about the inner binary corresponding to the circularization radius and
v$_{\rm orb,c}$ is the orbital velocity at $a_{\rm c}$.  The distance
from the center of mass corresponding to the outer tertiary companion
defined by the Roche lobe is given by Equation 2 in
\citep{1983ApJ...268..368E}.  Combining Equation 2 in \citet{1983ApJ...268..368E} (with
mass ratio q $=$ m$_{\rm 3}$/(m$_{\rm 1} +$m$_{\rm 2}$)) with
Equation~\ref{eqn:specangmom1}, we can solve for the circularization
radius as a function of a$_{\rm out}$ and the assumed stellar masses:
\begin{equation}
\label{eqn:ac}
a_{\rm c} = a_{\rm out}(1 - R_{\rm L}).
\end{equation}
In order for a circumbinary disk to form around the inner binary, we
require that a$_{\rm in} <$ a$_{\rm c}$.

Figure~\ref{fig:fig2} shows the parameter space in the P$_{\rm
  out}$-P$_{\rm in}$-plane for Binary 7782.  We assume initial
component masses of $m_{\rm 1} = 1.1$ M$_{\rm \odot}$ and $m_{\rm 2}
= 0.9$ M$_{\rm \odot}$ for the inner binary components, and $m_{\rm 3}
= 1.4 $M$_{\rm \odot}$ for the outer tertiary.  We compare the
circularization radius to the semi-major axis of the inner binary, for
which we require $a_{\rm c} > a_{\rm in}$, after folding in all
constraints from the requirements for dynamical stability (listed in
the caption of Figure~\ref{fig:fig2}), an outer tertiary that is Roche
lobe-filling, and a dynamically hard outer tertiary orbit.  Note that
the range of plotted orbital periods P$_{\rm in}$ corresponding to a
contact state for the inner binary lies outside the range of plotted
values for $P_{\rm in}$ (for components with radii of roughly 1
R$_{\odot}$), since it does not contribute significantly to
constraining the outer orbital properties of a hypothesized outer WD
tertiary).  The thick horizontal solid red line shows the allowed range
of outer semi-major axes, after folding in all of the aforementioned
criteria.  As is clear, this makes a relatively narrow prediction for
the allowed ranges of outer tertiary orbits, namely 2.2 $\times$
10$^{2}$ days $\le$ P$_{\rm out}$ $\le$ 1.1 $\times$ 10$^3$ days, for
our assumed final donor mass.

\begin{figure}[ht!]
\plotone{fig2.eps}
\caption{Parameter space in the P$_{\rm out}$-P$_{\rm in}$-plane
  allowed for the hypothetical outer tertiary orbit of Binary 7782.
  The solid diagonal black line shows the period corresponding to the
  circularization radius a$_{\rm c}$ for the mass transfer stream
  coming from the outer star (i.e., at the onset of mass transfer), assuming a mass ratio of q $=$ 0.7 and component masses 
  of m$_{\rm 1} =$ 0.9 M$_{\odot}$, m$_{\rm 2} =$ 1.1 M$_{\odot}$ and 
  m$_{\rm 3} =$ 1.4 M$_{\odot}$. \simon{what is the purpose of extending q to 0.1 and
    10? - NL2: I was originally thinking that q changes during MT, but I have now removed the other lines and just left the one corresponding to our test case.} We assume initial component masses of m$_{\rm 1}$ = 1.1
  M$_{\rm \odot}$ and m$_{\rm 2}$ = 0.9 M$_{\rm \odot}$ for the inner
  binary components, and m$_{\rm 3}$ is computed for the outer
  tertiary according to our assumed mass ratio (with our fiducial case
  corresponding to q $=$ 0.7).  We assume completely conservative mass
  transfer for this exercise, and a final mass for the outer tertiary
  of 0.6 M$_{\rm \odot}$ once it has become a WD.  The dashed diagonal
  black line shows a rough criterion for dynamical stability in the
  triple, approximately following \citet{mardling99} (i.e., a$_{\rm
    in} <$ 0.1a$_{\rm out}$ is required for long-term dynamical
  stability in equal-mass co-planar triples).  The vertical solid red
  lines show the outer orbital periods corresponding to the hard-soft
  boundary assuming central velocity dispersions of $\sigma =$ 1, 5
  and 10 km s$^{-1}$.  The vertical dashed black lines show the
  maximum outer orbital period P$_{\rm out}$ for which the outer
  tertiary companion is Roche lobe-filling, assuming a stellar radius
  of R$_{\rm 3} =$ 200 R$_{\odot}$ (which corresponds to the maximum stellar radius reached on the AGB for the range of tertiary masses of interest to us.  The horizontal dashed red line shows the observed
  orbital period for Binary 7782, using its observed orbital period
  and our assumed final inner companion masses (i.e., m$_{\rm 1} =$
  m$_{\rm 2} =$ 1.4 M$_{\odot}$).  Finally, the thick solid horizontal
  red line shows the parameter space for P$_{\rm out}$ allowed after
  considering all of the aforementioned criteria.  \simon{I think that
    the figure needs work. I like putting the parameter space in terms
    of orbital periods, but the q=0.1 and q=10 constraints seems
    not very practical. The red labels are not readable, and I am not
    really sure how to read the figure. What about the alternative in
    the next figure, which shows tertiary mass vs. orbital period? - NL2: I have now edited this figure, removed the other q curves, increased font size, etc., but am okay with it if you want to remove it for something else.}
\label{fig:fig2}}
\end{figure}

%NL2: Can we also plot the radii at the terminal-age RGB and AGB, or some maximum radius during each evolutionary phase?  
%I know this will make the plot busier, but these some more relevant to me for mass transfer than the zero-age RGB and 
%AGB radii.  What do you think?
\begin{figure}[ht!]
  \includegraphics[width=\columnwidth]{fig_minimumstablesize.pdf}
\caption{Orbital separation as a function of the mass of the
  Roche-lobe filling outer star. The red curves give the stellar radius r$_{\rm 3}$ at the
  first giant branch (thick curve) and at the AGB (thin curve). The
  blue curves give the orbital separation a$_{\rm RLOF}$ at the moment this giant fills its
  Roche-lobe assuming a circular orbit and a total inner binary mass
  of 2.0\,\MSun. The green curves give, under these conditions, the
  maximum orbital separation for a stable inner binary, or a$_{\rm stable}$.
\label{fig:tertiarymass_vs_size}}
\end{figure}

%NL2: I tried to fill in the text below, but please go through and edit it as you see fit.  Also, we need to decide which figures we are going to keep, since I am guessing there is a limit for ApJL.  This of course relates to the current Figure 2, which you may feel should be removed and would be fine with me.
In Fig.\,\ref{fig:tertiarymass_vs_size} we present several characteristic sizes and orbital separations for the problem at hand.  
The red curves show the stellar radius upon reaching the first giant branch (thick curve) and the asymptotic 
giant branch (thin curves).  The blue curves show the orbital separation at the moment of Roche-lobe overflow for each of the aforementioned giants, 
assuming a circular orbit and a total inner binary mass of $0.9 + 1.1$ M$_{\odot}$ $=$ 2.0 M$_{\odot}$.  The green curves show 
for these assumptions the maximum orbital separation for a stable inner binary pair.    

\section{Numerical Simulations} \label{sims}

We perform simulations of a triple star system for which the outer
star overfills its Roche lobe while the inner binary remains
detached. The calculations start by evolving the three stars to the
same age, which is selected such that the outer-most star fills its
Roche lobe.  First order constraints for the initial conditions are
derived in the previous \S. In the following two sections we describe
how we setup these simulations and then discuss the results. The
calculations are performed using the

\subsection{Setting-up the simulations}

We adopt initial masses of $m_{\rm 1} = 1.1$ M$_{\rm \odot}$ and
$m_{\rm 2} = 0.9$ M$_{\rm \odot}$ for the inner binary components, and
between $m_{\rm 3} = 1.2$ and $m_{\rm 3} = 1.4$ M$_{\rm \odot}$ for
the tertiary star.  We evolve the tertiary star using the MESA
stellar-evolution code \cite{2011ApJS..192....3P} to a radius of about
1\,au, at which point we assume it to overfill it's Roche lobe.  By
this time the star has lost 0.13\,\MSun\, via a stellar wind.

The stellar-evolution model, including the structure, temperature and
composition profiles, are turned into a smoothed-particles
representation using the module {\tt StellarModelInSPH} in AMUSE (see
chapter 4 in \cite{AMUSE}).  We follow the same procedure as described
in \cite{2014MNRAS.438.1909D} for simulating the future of the triple
system $\chi$ Tau (HD 97131) in which the outer-most star overfills its
Roche lobe and trnsfers mass to an inner binary.  After generating the hydrodynamical
representation of the donor star we replace the stellar core by a
point mass to prevent the majority of the resolution to be confined in
the star's central regions.  In a following step we relax the star
using the hydrodynamics solver. This relaxation process is realized in
100 steps during which we reduce the velocity dispersion of individual
SPH particles to a glasseas structure \citep[see, for example, \S\,3.3
  on page 40 in][]{1994astro.ph.10043W}. During this procedure the
gaseous envelope of the star tends to expand by about 20\%.  To
detemrine the radius of the evolving star we calculate Lagrangian
radii and use the distance to the stellar center which contains 90\%
of it's mass. From this 90\% mass-radius relation we obtain the stellar radius,
and match it with the Roche-lobe of the outer orbit.

With these parameters the orbital separation of the outer binary
becomes $\sim 3.86$\,au, and we adopt the outer orbit to be circular.
We choose a relative initial inclination between the inner and the
outer orbits of $i = 0^\circ$, $i = 9^\circ$, $i = 90^\circ$ and $i =
180^\circ$.  The first two choices are motivated by observed low-mass stellar
triples composed of low-mass stars which have been observed to favor
co-planar configurations \citep{2018ApJ...854...44M}. To test the effect of highly
inclined and retrogade orbits we include the $i = 90^\circ$ and $i =
180^\circ$ cases.  In Fig.\,\ref{fig:topview_at_t0} we present a top
view of the triple system in our simulations at the onset of Roche-lobe overflow.

\begin{figure}[ht!]
  \includegraphics[width=\columnwidth]{fig_t0_N80000_M012MSun1109MSun_a02au_e00_inc9deg.pdf}
\caption{Top view of the simulated triple system in which the outer
  star over-fills its Roche-lobe (represented by 80000\,sph particles
  and a core particle of 0.4\,\MSun (blue bullet). The two companion
  stars are represented as red bullets (to the right).
\label{fig:topview_at_t0}}
\end{figure}

Roche-lobe overflow in triples is modeled using a coupled integrator
to follow the complex hydrodynamics of mass transfer from the
Roche-lobe filling outer star to the inner binary, while keeping track
of the gravitational dynamics of the stars.  The equations of motion
of the inner binary are solved using the symplectic direct N-body
integrator \texttt{Huayno} \citep{2012NewA...17..711P}. The
hydrodynamics are performed with the smoothed-particles hydrodynamics
code \texttt{Gadget2} \citep{2000ascl.soft03001S}, using an adiabatic
equation of state.  The two inner binary stars are treated as point
masses, but we allow them to accrete mass and angular momentum from
the gas liberated by the outer star.  This is realized using
sink-particles that co-move with the mass points in the gravity
code. While the inner two stars accrete mass, they also accrete the
corresponding amount of angular momentum from the gas (see chapter 5
in \cite{AMUSE}).  The N-body integrator correctly accounts for this.  For the radius of the sink particles we adopt $10
R_\odot$ for both stars.

The N-body code as well as the hydrodynamics solver operate using
their own internal time-steps. The coupling between the two codes is
realized using the \texttt{Bridge} method in the AMUSE framework
\citep[see Sect.\.4.3.1 in][]{2013CoPhC.183..456P}.  This coupled
integrator is based on the splitting of the Hamiltonian, much in the
same way as is done with two different gravity solvers by
\cite{2007PASJ...59.1095F}. With the adopted scheme, the
hydrodynamical solver is affected by the gravitational potential of
its own particles, as well as the gravitational potential of the inner
binary. The hydrodynamics, in particular the interaction with the gas,
in turn affects the orbits of the two inner stars. With
\texttt{Bridge} we realize a second order coupling between the
gravitational dynamics and the hydrodynamics.  The interval at which
the gravity and hydrodynamics interact via \texttt{Bridge} depends on
the parameters of the system we study, but typically we achieve
converged solutions when this time step is about a fraction of 1/100
of the inner binary orbital period.

\section{Results} \label{results}

In fig.\,\ref{fig:topview_at_t1000day} we show the same triple after
1000\,days of evolution.
%NL2:  We are labeling the figures differently in the text.  You label with ``fig.\,\", whereas I use ``Figure~''.  It doesn't matter to me which we use, just thought I'd highlight this so we come back to it right before submitting.

\cite{2018arXiv181208175M}


\begin{figure}[ht!]
  \includegraphics[width=0.49\columnwidth]{lagrange_points_and_sph.pdf}
~  \includegraphics[width=0.51\columnwidth]{fig_t600_N80000_M012MSun1109MSun_a02au_e00_inc9deg.pdf}
\caption{Same as Fig.\,\ref{fig:topview_at_t0} but at an age of $t
  \simeq 996days$. Left panel shows the equipotential surface of the
  triple overplotted with the gas distribution, the right panel shows
  just the gas and the stars as bullets.
\label{fig:topview_at_t1000day}}
\end{figure}

In fig.\,\ref{fig:mass_vs_semimajor_axis}

\begin{figure}[ht!]
  \includegraphics[width=\columnwidth]{fig_orbital_evolution_for_12MSun.pdf}
  \includegraphics[width=\columnwidth]{fig_orbital_evolution_for_14MSun.pdf}
  \caption{Evolution of the total mass of the inner binary as a
    function of the orbital separation for four calculations with
    somewhat different initial conditions (see legend).  The initial
    binary for the orange curve is presented in
    fig.\,\ref{fig:topview_at_t0}, and in
    fig.\,\ref{fig:topview_at_t1000day} we present the final
    conditions.
\label{fig:mass_vs_semimajor_axis}}
\end{figure}

The orbital evolution of the inner binary circumreferenced by a disk
tends to evove by the accretion of mass and angular momentum. As a
consequence, the two stars in the binary accumulate mass, and their
orbit shrinks. The orbital evolution can be approximated, to first order, by:
\begin{equation}
  {d\ln a \over dt}   = {-d\ln m_{\rm in} \over dt},
\end{equation}
where $a$ and $m_{\rm in}$ are the semimajor-axis and total mass in
the inner binary.  In the simulations we measure ${d\ln a \over dt}
\simeq -0.0843$\,Myr$^{-1}$.
%NL2:  The above equation seems incorrect to me, dimensionally.  It can be fixed by saying the ``specific mass'' and ``dimensionless semi-major axis'', I suppose.  But then I'd define how you make those parameters dimensionless.

With the accretion of mass, both stars in the inner binary also accrete
angular momentum.  By the end of the simulation the spins of the two
blue stragglers are aligned along the orbital angular momeumt axis with
an angle of $90.0^\circ$ for the primary star and $93.4^\circ$ for the
secondary star with respect to the argument of pericenter of the inner
orbit.  By the end of the simulations the spin of the primary is about
50.5 rotations per day, and 41.5 rotations per day for the secondary star. But by this
time both stars have accreted only a small portion of mass, and we expect that
the final spins of both stars will be close to break-up (ignoring any spin-down from magnetic braking effects).
%NL2: I added the bit about magnetic braking above but it needs to be worded better, or removed.  Will come back to it, and feel free to edit it yourself as well..

\section{Discussion} \label{sect:discussion}

In this Letter, we consider the formation of double blue straggler compct binaries formed 
from mass transfer from an outer tertiary companion.  As illustrated via SPH simulations, 
the mass transfer stream forms a circumbinary 
disk, from which the inner binary components accrete, driving the inner binary toward a 
mass ratio close to unity.  Our simulations suggest that the inner binary orbital separation 
should decrease in response to mass accretion, for the assumed initial conditions adopted here.  A more 
detailed study of the relevant parameter is worth performing in future work.  For example, under what circumstances 
might the inner binary pair merge, forming a much more massive BS?  Conversely, if the orbital 
separation of the inner binary \textit{increases} in response to mass transfer, this could 
cause the triple to become dynamically unstable, subsequently undergoing a chaotic interaction, most likely 
resulting in the ejection of the lowest mass object in the system.  If one of the two BSs formed in 
this scenario are ejected, this would leave behind a BS binary with a WD companion.  Such systems 
are assumed to solely be the result of mass transfer in binary systems, but the results presented here 
suggest that such an assumption should be regarded with caution.

In order to study the T-tauri binaries V4046 Sgr and DQ Tau, 
\cite{2011MNRAS.413.2679D} perform a series of 2D hydrodynamical
simulations of circumbinary disks.  These authors studied the two observed
T-tauri systems V4046 Sgr and DQ~Tau, to which we compare our results here.

For V4046 Sgr, for which the two stars have comparable masses as in
our calculation in an ciruclar orbit with a period of only 2.4 days
they find that the inner binary accretes at a rate of $\sim
0.028$\,\MSun/Myr.  For DQ~Tau, which is composed of lower-mass stars
($m_1 = m_2 \simeq 0.55$\,\MSun) in an eccentric ($e\simeq 0.556$)
orbit of $\sim 15.8$\,days they find an accretion rate onto the inner
binary of $\sim 0.027$\,\MSun/Myr.  These values are in the same range
as in our calculations, which results in an accretion rate of the inner
binary of 0.027--0.058\,\MSun/Myr (i.e., the average measured over a period of
about 3000 days in our simulations).  The highest accretion rate we obtain for the
simulations with an eccentric inner orbit.  Here we applied a small
correction to compensate for the larger accretion radii adopted for
the two stars of 10\,\RSun\, compared to 6\,\RSun\, in
\cite{2011MNRAS.413.2679D}. Interestingly, however,
\cite{2011MNRAS.413.2679D} find that the primary star in V4046 Sgr
accretes at an 8\% higher rate than the secondary star, whereas in our
case the secondary star accretes at a higher rate than the primary
star by about 1\% to 12\%.  Higher accretion rates in the secondary
star are realized for eccentric and retrogade inner orbits.
%NL2:  Huh.  This is interesting.  Is the physical reason for it that apocenter for the secondary tends to be closer to the circumbinary disk?  I don't get why retrograde wins though, since naively I'd have thought prograde should lead to lower relative velocities between the secondary and the gas flow, and hence to higher accretion rates.  What am I missing?  Is there momentum cancellation via colliding gas, shocks, etc. I am missing somewhere?



\section{Summary} \label{sect:conclusions}

In this Letter, we have proposed a formation scenario for double BS
equal-mass compact binaries, as observed for Binary 7782 in the old
open cluster NGC 188.  The proposed scenario involves mass transfer
from an evolved outer tertiary companion, which is accreted by the
inner binary via a circumbinary disk.  Our scenario makes several
predictions for the observed properties of a hypothetical outer triple
companion, now a WD.  These are:

\begin{enumerate}

\item For the predicted outer tertiary orbit, the present-day
  semi-major axis should lie between 220 days $\lesssim$ P$_{\rm out}$
  $\lesssim$ 1100 days, assuming initial masses for the inner binary
  components of $m_{\rm 1} = 1.1$ M$_{\odot}$ and $m_{\rm 2} = 0.9$
  M$_{\odot}$ and an initial outer tertiary mass of $m_{\rm 3} = 1.4
  $M$_{\odot}$.

\item Larger final WD masses, and hence larger core masses for the donor at
  the time of mass transfer, should correspond to larger final outer
  orbital periods for the tertiary.  This is because the Roche radius
  is larger for larger outer orbital periods, such that the donor must
  evolve to larger radii, and hence core masses, before the onset of
  mass transfer.

\item For the inner binary, the rotational axes of both the BSs should
  be aligned with each other and the orbital plane of the outer
  tertiary WD.  This is because accretion onto the BS progenitors
  proceeds via an accretion disk, that forms at the circularization
  radius and that has an orbital plane aligned with that of the outer
  tertiary.

\item The BSs in the inner binary should have roughly equal masses,
  independent of their initial masses.  This is because it is the
  lowest mass object that typically accretes the fastest, since its
  orbital velocity and distance relative to the circumbinary disk is
  typically the lowest
  \citep[e.g.][]{1995MNRAS.277.1491K,1997MNRAS.285...33B,2000MNRAS.314...33B}.
  This quickly brings the mass ratio toward unity.  This predicts that
  the initially lower mass MS star should accrete the most, and should
  hence be enriched more by any accreted material.  This could be
  observable in the surface layers of a radiative star.  If the donor
  is an RGB star, the accretor will be enriched in mostly carbon,
  oxygen and helium.  If the donor is an AGB star, it will be enriched
  in mostly s-process elements.

\item Twin BSs in compact binaries formed from the mechanism proposed here 
  should be more frequent in younger clusters with ages $\lesssim$ 4-6
  Gyr.  This is because clusters with a main-sequence turn-off mass
  $\lesssim$ 1.2 M$_{\odot}$ have convective envelopes
  \citep[e.g.][]{1991ApJS...76...55I,2009pfer.book.....M}, and a radiative envelope for the
  donor in a mass transferring binary ensures stable accretion on to
  the accretor.  \simon{Can we quantify this?}

\end{enumerate}

We emphasize in closing that our choice for the initial mass of the outer tertiary is critical since it
ensures that the donor star during the mass transfer process will have
a radiative envelope \citep[e.g.][]{2009pfer.book.....M}.  In turn, this ensures
that the mass transfer will be maximally conservative, such that the
accretion stream will be maximally stable, accreting at a stable and
roughly constant rate \citep[e.g.][]{1991ApJS...76...55I}.

%\simon{Why would we try to sell already what we plan on doing. I'de prefer to keep this quet. - NL2: To be honest, the ideas came to me, seemed relevant, and I was more interested in if *you* thought they were interesting ideas worth pursuing.  If that is the case, then yes let's pursue it ourselves and remove this paragraph (now done). Happy to work on these ideas, and actually have been in the WD+WD case with a student but it is going very slowly.  I think this mechanism should tie in nicely with that project though, so I'd like to include you on that if the student gets what he needs to done... it might be a lot quicker for us to just do it though. :) }
%Finally, in future work, we intend to explore the implications of the
%mechanism proposed here for other types of stellar triples.  Our
%results highlight that specific theoretical predictions can often be
%made.  For example, if two white dwarfs form the inner compact binary,
%than circumbinary accretion could push one of them above the
%Chandrasekhar mass limit.  Upon explosion, its WD companion would be
%ejected at roughly the orbital velocity at the time of explosion,
%producing a hypervelocity WD.  This predicts hot hypervelocity WDs
%with masses very close to the Chandrasekhar mass limit for this
%scenario.  Similarly, if the two WDs are replaced with two neutron
%stars (NSs), then one NS could accrete to surpass the critical
%supernova limit, upon which time it explodes to become a black hole
%and receives a strong natal kick.  The NS companion would have a very
%comparable mass to the newly formed black hole progenitor at the time
%of explosion, and could potentially be used to constrain the maximum
%NS mass.  More work needs to be done here to evaluate whether or not
%there is any parameter space for the products of this scenario that
%will allow for a mass determination for the original NS companion in
%the putative compact NS-NS binary.

\acknowledgments

N.W.C.L. acknowledges support from a Kalbfleisch Fellowship at the
American Museum of Natural History.  SPZ would like to thank Norm
Murray and CITA for their hospitality during my long-term visit.  This
work was supported by the Netherlands Research School for Astronomy
(NOVA), NWO (grant \# 621.016.701 [LGM-II]).

\bibliographystyle{apj}
\bibliography{BSS7782}


%\input /home/spz/Latex/lib/bib/references
%\bibliographystyle{/home/spz/Latex/lib/styles/elsevier/elsarticle-num} 
%\bibliography{references}      


\end{document}
